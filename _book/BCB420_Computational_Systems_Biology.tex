\documentclass[]{book}
\usepackage{lmodern}
\usepackage{amssymb,amsmath}
\usepackage{ifxetex,ifluatex}
\usepackage{fixltx2e} % provides \textsubscript
\ifnum 0\ifxetex 1\fi\ifluatex 1\fi=0 % if pdftex
  \usepackage[T1]{fontenc}
  \usepackage[utf8]{inputenc}
\else % if luatex or xelatex
  \ifxetex
    \usepackage{mathspec}
  \else
    \usepackage{fontspec}
  \fi
  \defaultfontfeatures{Ligatures=TeX,Scale=MatchLowercase}
\fi
% use upquote if available, for straight quotes in verbatim environments
\IfFileExists{upquote.sty}{\usepackage{upquote}}{}
% use microtype if available
\IfFileExists{microtype.sty}{%
\usepackage{microtype}
\UseMicrotypeSet[protrusion]{basicmath} % disable protrusion for tt fonts
}{}
\usepackage{hyperref}
\hypersetup{unicode=true,
            pdftitle={BCB420 - Computational System Biology},
            pdfauthor={Boris Steipe},
            pdfborder={0 0 0},
            breaklinks=true}
\urlstyle{same}  % don't use monospace font for urls
\usepackage{natbib}
\bibliographystyle{apalike}
\usepackage{color}
\usepackage{fancyvrb}
\newcommand{\VerbBar}{|}
\newcommand{\VERB}{\Verb[commandchars=\\\{\}]}
\DefineVerbatimEnvironment{Highlighting}{Verbatim}{commandchars=\\\{\}}
% Add ',fontsize=\small' for more characters per line
\usepackage{framed}
\definecolor{shadecolor}{RGB}{248,248,248}
\newenvironment{Shaded}{\begin{snugshade}}{\end{snugshade}}
\newcommand{\KeywordTok}[1]{\textcolor[rgb]{0.13,0.29,0.53}{\textbf{#1}}}
\newcommand{\DataTypeTok}[1]{\textcolor[rgb]{0.13,0.29,0.53}{#1}}
\newcommand{\DecValTok}[1]{\textcolor[rgb]{0.00,0.00,0.81}{#1}}
\newcommand{\BaseNTok}[1]{\textcolor[rgb]{0.00,0.00,0.81}{#1}}
\newcommand{\FloatTok}[1]{\textcolor[rgb]{0.00,0.00,0.81}{#1}}
\newcommand{\ConstantTok}[1]{\textcolor[rgb]{0.00,0.00,0.00}{#1}}
\newcommand{\CharTok}[1]{\textcolor[rgb]{0.31,0.60,0.02}{#1}}
\newcommand{\SpecialCharTok}[1]{\textcolor[rgb]{0.00,0.00,0.00}{#1}}
\newcommand{\StringTok}[1]{\textcolor[rgb]{0.31,0.60,0.02}{#1}}
\newcommand{\VerbatimStringTok}[1]{\textcolor[rgb]{0.31,0.60,0.02}{#1}}
\newcommand{\SpecialStringTok}[1]{\textcolor[rgb]{0.31,0.60,0.02}{#1}}
\newcommand{\ImportTok}[1]{#1}
\newcommand{\CommentTok}[1]{\textcolor[rgb]{0.56,0.35,0.01}{\textit{#1}}}
\newcommand{\DocumentationTok}[1]{\textcolor[rgb]{0.56,0.35,0.01}{\textbf{\textit{#1}}}}
\newcommand{\AnnotationTok}[1]{\textcolor[rgb]{0.56,0.35,0.01}{\textbf{\textit{#1}}}}
\newcommand{\CommentVarTok}[1]{\textcolor[rgb]{0.56,0.35,0.01}{\textbf{\textit{#1}}}}
\newcommand{\OtherTok}[1]{\textcolor[rgb]{0.56,0.35,0.01}{#1}}
\newcommand{\FunctionTok}[1]{\textcolor[rgb]{0.00,0.00,0.00}{#1}}
\newcommand{\VariableTok}[1]{\textcolor[rgb]{0.00,0.00,0.00}{#1}}
\newcommand{\ControlFlowTok}[1]{\textcolor[rgb]{0.13,0.29,0.53}{\textbf{#1}}}
\newcommand{\OperatorTok}[1]{\textcolor[rgb]{0.81,0.36,0.00}{\textbf{#1}}}
\newcommand{\BuiltInTok}[1]{#1}
\newcommand{\ExtensionTok}[1]{#1}
\newcommand{\PreprocessorTok}[1]{\textcolor[rgb]{0.56,0.35,0.01}{\textit{#1}}}
\newcommand{\AttributeTok}[1]{\textcolor[rgb]{0.77,0.63,0.00}{#1}}
\newcommand{\RegionMarkerTok}[1]{#1}
\newcommand{\InformationTok}[1]{\textcolor[rgb]{0.56,0.35,0.01}{\textbf{\textit{#1}}}}
\newcommand{\WarningTok}[1]{\textcolor[rgb]{0.56,0.35,0.01}{\textbf{\textit{#1}}}}
\newcommand{\AlertTok}[1]{\textcolor[rgb]{0.94,0.16,0.16}{#1}}
\newcommand{\ErrorTok}[1]{\textcolor[rgb]{0.64,0.00,0.00}{\textbf{#1}}}
\newcommand{\NormalTok}[1]{#1}
\usepackage{longtable,booktabs}
\usepackage{graphicx}
% grffile has become a legacy package: https://ctan.org/pkg/grffile
\IfFileExists{grffile.sty}{%
\usepackage{grffile}
}{}
\makeatletter
\def\maxwidth{\ifdim\Gin@nat@width>\linewidth\linewidth\else\Gin@nat@width\fi}
\def\maxheight{\ifdim\Gin@nat@height>\textheight\textheight\else\Gin@nat@height\fi}
\makeatother
% Scale images if necessary, so that they will not overflow the page
% margins by default, and it is still possible to overwrite the defaults
% using explicit options in \includegraphics[width, height, ...]{}
\setkeys{Gin}{width=\maxwidth,height=\maxheight,keepaspectratio}
\IfFileExists{parskip.sty}{%
\usepackage{parskip}
}{% else
\setlength{\parindent}{0pt}
\setlength{\parskip}{6pt plus 2pt minus 1pt}
}
\setlength{\emergencystretch}{3em}  % prevent overfull lines
\providecommand{\tightlist}{%
  \setlength{\itemsep}{0pt}\setlength{\parskip}{0pt}}
\setcounter{secnumdepth}{5}
% Redefines (sub)paragraphs to behave more like sections
\ifx\paragraph\undefined\else
\let\oldparagraph\paragraph
\renewcommand{\paragraph}[1]{\oldparagraph{#1}\mbox{}}
\fi
\ifx\subparagraph\undefined\else
\let\oldsubparagraph\subparagraph
\renewcommand{\subparagraph}[1]{\oldsubparagraph{#1}\mbox{}}
\fi

%%% Use protect on footnotes to avoid problems with footnotes in titles
\let\rmarkdownfootnote\footnote%
\def\footnote{\protect\rmarkdownfootnote}

%%% Change title format to be more compact
\usepackage{titling}

% Create subtitle command for use in maketitle
\providecommand{\subtitle}[1]{
  \posttitle{
    \begin{center}\large#1\end{center}
    }
}

\setlength{\droptitle}{-2em}

  \title{BCB420 - Computational System Biology}
    \pretitle{\vspace{\droptitle}\centering\huge}
  \posttitle{\par}
    \author{Boris Steipe}
    \preauthor{\centering\large\emph}
  \postauthor{\par}
      \predate{\centering\large\emph}
  \postdate{\par}
    \date{2019-01-04}

\usepackage{booktabs}
\usepackage{amsthm}
\makeatletter
\def\thm@space@setup{%
  \thm@preskip=8pt plus 2pt minus 4pt
  \thm@postskip=\thm@preskip
}
\makeatother

\begin{document}
\maketitle

{
\setcounter{tocdepth}{1}
\tableofcontents
}
\chapter{Wiki - Editing}\label{wiki}

(Wiki editing, namespaces; user page setup; copyright, a Course Journal
stub page, and an insights! stub page)

\section{Abstract:}\label{abstract}

This will likely be the first learning unit you work with, since your
Course Journal will be kept on a Wiki, as well as all other
deliverables. This unit includes an introduction to authoring Wikitext
and the structure of Wikis, in particular how different pages live in
separate ``Namespaces''. The unit also covers the standard markup
conventions - ``Wikitext markup'' - the same conventions that are used
on Wikipedia - as well as some extensions that are specific to our
Course- and Student Wiki. We also discuss page categories that help keep
a Wiki organized, licensing under a Creative Commons Attribution
license, and how to add licenses and other page components through
template codes.

\section{Objectives:}\label{objectives}

\begin{itemize}
\tightlist
\item
  Provide an introduction to Wiki principles and Wikitext markup.
\item
  Create first pages on your own on the Student Wiki.
\item
  Learn about copyright, why we use Creative Commons licenses for the
  Student Wiki and how to place a license tag.
\end{itemize}

\section{Outcomes:}\label{outcomes}

\begin{itemize}
\tightlist
\item
  You are competent with basic Wiki markup and the extensions on this
  Wiki.
\item
  You can create pages and add them to categories while taking care to
  create them into your own user space.
\item
  You have created your own user page on the Student Wiki and added
  contents.
\item
  You have created page stubs for a Course Journal and an insights!
  page.
\end{itemize}

\section{Deliverables:}\label{deliverables}

\begin{itemize}
\tightlist
\item
  Specified as ``Tasks'': There are no general deliverables for this
  unit; specific deliverables are described in the ``Task'' sections.
\end{itemize}

\begin{center}\rule{0.5\linewidth}{\linethickness}\end{center}

Collaboration is a common theme for modern lab work and a Wiki is a
great way to share and seamlessly update information in groups - or just
for yourself. Probably the most sophisticated Wiki software is
MediaWiki, a set of PHP scripts that is under continuous development by
the Wikimedia foundation; it is the same software that runs Wikipedia.
This is open source, free software that is easy to install, is well
documented and requires very little resources other than a machine that
runs a MySQL database server and an Apache Webserver. Numerous
extensions exist (and extensions are not hard to write); they enhance
the already rich functionality. But let's start with small steps. You
should by now have a user account on the Student Wiki, and I have
configured that Wiki so that * only logged in users can view the pages;
but \ldots{} * all logged in users can create and edit (most) pages at
will. * This means you could edit pages that don't ``belong'' to you.
Respect the ``House Rules'' and don't edit other's things without
permission, even if you can think of a particularly witty comment or
hilarious prank. If you want to comment on a page: every page has an
associated ``Discussion'' page that you can freely edit. Remember to
``sign your name'' to discussion entries.

\subsection{Task:}\label{task}

\begin{enumerate}
\def\labelenumi{\arabic{enumi}.}
\tightlist
\item
  Access the Student Wiki;
\item
  log in and navigate to your user page. (Your user page is linked to
  your name that appears at the top of every Wikipage once you are
  logged in.)
\item
  Create / edit the page, try out and experiment with the Wikitext
  syntax that this unit covers as you read about the different elements.
\item
  Work through the contents below.
\end{enumerate}

For more extensive formatting help see:
\url{http://meta.wikimedia.org/wiki/Help:Editing} For Math markup see:
\url{http://meta.wikimedia.org/wiki/Help:Formula}

\section{The Wiki concept}\label{the-wiki-concept}

Wiki sites are collections of Web pages that allow you to view, edit and
create pages from your browser, there is no need for special technology
and basic editing is simple and intuitive with ``Wikitext markup''. The
basic workflow of Wikis is straightforward: * Register an account and
browse the site. * Whenever you find something that you can improve,
edit it. * Whenever you find something that you would like to comment
on, click on the ``discussion'' tab and share your views. * If you are
interested in what becomes of your edits or the discussion, click on the
``watch'' tab, and the page will be added to a list of bookmarks to
pages you are ``watching''. (You can even generate an RSS feed for
recent changes or new pages). * No e-mail, no obligations. Do what you
like, when you like, what you can. * Editing on the Course Wiki is only
enabled for instructors. However you can freely edit all pages on the
Student Wiki, once you have an account.

\section{Editing basics}\label{editing-basics}

\subsection{Start editing}\label{start-editing}

To start editing a Wiki page, click on the ``Edit'' link at one of its
edges. This will bring you to the edit page: a page with a text box
containing the wikitext: the editable source code from which the server
produces the webpage.

\subsection{Preview before saving}\label{preview-before-saving}

When you have finished, press Show preview to see how your changes will
look. Repeat the edit/preview process until you are satisfied, then
click Save and your changes will be immediately applied to the article
and accessible on the Web. They will also be stored in the main database
for as long as the Wiki exists. Thus it is always possible to get back
earlier versions of pages - back to the very first edit.

\subsection{Basic text formatting}\label{basic-text-formatting}

Here are some examples of the markup of Wikitext. It is not the same as
HTML markup, however some HTML markup will work. In particular, the Wiki
applies styles through CSS technology (Cascading Style Sheets) and thus
HTML tags can be used to apply consistent styles to individual page
elements. Javascript won't work. What it looks like What you type You
can emphasize text by putting two apostrophes on each side. Three
apostrophes will emphasize it strongly. Five apostrophes is even
stronger. You can `'emphasize text'`by putting two apostrophes on each
side. Three apostrophes will emphasize it'`'strongly'`'. Five
apostrophes is'`''`even stronger'`'''.

A single newline has no effect on the layout. But an empty line starts a
new paragraph. A single newline has no effect on the layout.

But an empty line starts a new paragraph.

You can break lines without starting a new paragraph. You can break
lines without starting a new paragraph.

You can format text in a monospace font with a dashed box around it
either by marking it with the HTML \textless{}pre\textgreater{} tag, or
by putting a blank space at the beginning of a line. Example.

(This may not be very useful beyond the types of examples we show here,
but it is a frequent source of confusion, when you find your text marked
up this way by accident) You can format text in a monospace font with a
dashed box around it either by marking it with the HTML
\textless{}pre\textgreater{}" tag, or by putting a blank space at the
beginning of a line.

Example.

(This may not be very useful beyond the types of examples we show here,
but it is a frequent source of confusion, when you find your text marked
up this way by accident)

Other special characters at the beginning of a line include: bulleted
list numbered list term and definition Other special characters at the
beginning of a line include:

\begin{itemize}
\tightlist
\item
  bulleted list
\end{itemize}

numbered list

\begin{description}
\tightlist
\item[; term]
and definition
\end{description}

You should ``sign'' your comments on discussion pages: Three tildes
gives your user name - Boris (talk) Four tildes: user name plus
date/time - Boris (talk) 22:18, 27 December 2012 (EST) Five tildes:
date/time alone - 22:18, 27 December 2012 (EST) You should ``sign'' your
comments on discusion pages: : Three tildes gives your user name -
\textsubscript{\textasciitilde{}} : Four tildes: user name plus
date/time - \textasciitilde{}\textsubscript{\textasciitilde{}} : Five
tildes: date/time alone -
\textsubscript{\textasciitilde{}}\textasciitilde{}\textasciitilde{}

Use normal HTML character codes for special characters, or use Unicode.
For example: \textgreater{} \textless{} \& ° Å Ä ü → Use normal HTML
character codes for special characters, or use Unicode. For example:
\textgreater{} \textless{} \& ° Å Ä ü →

You can use HTML tags, too, if you want. Some useful ways to use HTML:
Put text in a typewriter font. The same font is generally used for
computer code. Strike out or underline text, or write it in small caps.
Superscripts and subscripts: x2, x2 Invisible comments that only appear
while editing the page. You can use HTML tags, too, if you want. Some
useful ways to use HTML:

Put text in a typewriter font. The same font is generally used for
computer code.

Strike out or underline text, or write it \textsc{ in small caps}.

Superscripts and subscripts: x2, x2

Invisible comments that only appear while editing the page.

For a list of HTML tags that are allowed, see HTML in wikitext. I tend
to use Wiki-markup when I'm in a hurry, but use the HTML tag whenever I
can't remember a Wiki-tag. It really doesn't make a difference. However:
I never use Wiki-table markup. I find it less intuitive than HTML
markup, more difficult to debug, and there's really no point in
remembering both types of markup given that one really needs to be
comfortable with HTML tables anyway.

\subsection{Links}\label{links}

You will often want to make clickable links to other pages. What it
looks like What you type Here's a link to a page named Sandbox. You can
even say Sandboxes and the link will show up right. You can put
formatting around a link. Example: Sandbox. Here's a link to a page
named {[}{[}Sandbox{]}{]}. You can even say {[}{[}Sandbox{]}{]}es and
the link will show up right.

You can put formatting around a link. Example: `'{[}{[}Sandbox{]}{]}''.

You can link to an arbitrary piece of text with a piped link. Put the
link target first, then the pipe character ``\textbar{}'', then the link
text - as in this example. You can link an arbitrary piece of text with
a `'piped link''. Put the link target first, then the pipe character
``\textbar{}'', then the link text - as in {[}{[}Sandbox\textbar{} this
example{]}{]}.

You can make an external link to a Web page just by typing an URL,
e.g.~\url{http://igem.org} Or you can link arbitrary text: iGEM. (Note:
No ``\textbar{}'' for external links, URL and text are separated by a
blank, and only single square brackets!) Or you can generate a
footnote-like link: {[}1{]}. You can make an external link to a Web page
just by typing an URL, e.g.~\url{http://igem.org}

Or you can link arbitrary text: {[}\url{http://igem.org} iGEM{]}. (Note:
No ``\textbar{}'' for `''external''' links, URL and text are separated
by a blank, and only single square brackets!)

Or you can generate a footnote-like link: {[}\url{http://igem.org}{]}.

\textbf{Note}: remember: internal links (using {[}{[}\ldots{}{]}{]} tags
to link to pages on this Wiki) are separated from linked text with a
pipe character. External links (using {[}\ldots{}{]} tags to link to
pages elsewhere on the Internet) are separated from linked text with a
space character.

\subsection{Special syntax}\label{special-syntax}

Two special syntax items need to be mentioned: ``templates'' and ``magic
words'': \#\#\#\# Templates Templates are pieces of Wikitext that are
substituted where a code that links to them has been placed into a page.
For example, if you enter \{\{Lorem\}\} on a page, the ``Lorem ipsum
dolor sit amet \ldots{}'' placeholder text is inserted in place of that
code. Wikis make extensive use of templates. \#\#\#\# Magic words some
reserved ``magic''-words are replaced with dynamically created contents
when the page is rendered. For example \textbf{TOC} forces placing a
Table Of Contents at the position of this token rather than its default
position, while \textbf{NOTOC} suppresses creation of a Table Of
Contents on a page.

\subsection{Creating a new page}\label{creating-a-new-page}

To create a new page simply insert a link to a Wiki page, which has a
page name that does not exist yet. The link will appear in red (except
if you inadvertently used the name of a page that already exists), and
the new page will be created when you click on the link. Page names can
be long and contain blank spaces. Internally, all blank spaces are
converted to underscore characters, but you can use the page name
without underscores in links; the Wiki software translates this for you.

\subsection{Namespaces}\label{namespaces}

The Wiki maintains some pages in special collections, in so called
``namespaces''. This is useful, because the behaviour of the software
can be customized for different namespaces: for example you may be
allowed to edit in the main- and the user- namespace, but not in the
MediaWiki: namespace, where pages are held that affect the gears and
wires of the Wiki. Page names without a prefix live in the main space.
Some commonly used prefixes are: * User: - personal pages for user with
an account on the Wiki; * Talk: - discussion pages for comments on
pages, accessible via the ``Discussion'' tab; * Help: - this page for
example; * Template: - pages with reusable text. * Special: - pages that
implement special functionality (like login, user lists, or lists of
recently changed pages); * Category: - an index of pages that have been
given a common tag. This is a convenient way to access pages that are in
some way related;

\subsection{Categories}\label{categories}

Once your page has been edited, you can associate it with one or more
categories. Add the appropriate category tag by typing
{[}{[}Category:BCH441\_2013{]}{]} or {[}{[}Category:BCB410\_2013{]}{]}.
The page is then automatically linked from a page that collects all
pages with that category tag. I would prefer that you do not create new
categories; ask me if you feel a need for it.

\subsection{Creating a new section or subsection on a
page}\label{creating-a-new-section-or-subsection-on-a-page}

To create a section or subsection, simply insert a section header into
an existing section. Header levels are defined by the number of ``=''
characters before and after the header text. Click on an edit link of
this page to see example code. Once a page has more than two headings,
the Wiki automatically creates a table of contents. You can adjust the
position of the table of contents by typing the ``magic word''
\_\_TOC\_\_somewhere on your page (Note: double underscore), you can
also suppress having a table of contents created with\textbf{NOTOC}.

\subsection{Edit conflicts}\label{edit-conflicts}

If someone else makes an edit while you are making yours, the result is
an edit conflict. Many conflicts can be automatically resolved by the
Wiki. If it can't be resolved, however, you will need to resolve it
yourself. The Wiki gives you two text boxes, where the top one is the
other person's edit and the bottom one is your edit. Merge your edits
into the top edit box, which is the only one that will be saved.

\subsection{Reverting pages to a previous
state}\label{reverting-pages-to-a-previous-state}

Sometimes a page needs to be reverted to a previous state. Access the
page through a link to the Recent Changes special page:
Special:Recentchanges. Find the page you need to revert, click on the
hist link, click on the version you need and verify that it is the
correct one. Then click on the edit tab at the top and Save page. A new
version of the page is then created with the old text. Note that this
does not actually overwrite anything - all edits are archived in the
database.

\subsection{Special markup on this
Wiki}\label{special-markup-on-this-wiki}

Here are some special templates and extensions installed on this Wiki:

\begin{itemize}
\item
  Vertical space - The template code \{\{Vspace\}\} will insert a
  two-line high space to help structuring text.
\item
  References and footnotes - Enclosing text in
  \textless{}ref\textgreater{} \ldots{}
  \textless{}\textbackslash{}ref\textgreater{} tags will create a
  footnote reference and display the text wherever you place a tag on
  the page.
\item
  Syntax highlight - The GeSHI syntax highlighter extension is installed
  on this Wiki. Type: for (i in 1:5) \{ print(i\^{}2) \} \# 1 4 9 16 25
  \ldots{} to get: for (i in 1:5) \{ print(i\^{}2) \} \# 1 4 9 16 25
\item
  Pubmed Articles and abstracts - \{\{\#\url{pmid:15289071}\}\} This
  inserts the article information in a

  , formatted by Template:Pubmed. Steipe (2004) Consensus-based
  engineering of protein stability: from intrabodies to thermostable
  enzymes. Meth Enzymol 388:176-86. (pmid: 15289071) {[} PubMed {]}{[}
  DOI {]}
\end{itemize}

\{\{\#pmid: 15289071 \textbar{}Steipe2004\}\} * This formats the pubmed
parser output for the Cite extension; A footnote mark will be inserted
here{[}1{]} and the actual reference will appear beneath the section of
the page.

\begin{itemize}
\tightlist
\item
  Math markup - H = - \textbackslash{}sum\_\{i=0\}\^{}n p\_i
  \textbackslash{}log\_\{2\} p\_i\textless{}/math\textgreater{}

  \begin{itemize}
  \tightlist
  \item
    see: \url{http://meta.wikimedia.org/wiki/Help:Formula}
  \end{itemize}
\item
  ToDo items - Type:

  ToDo:
\item
  This \ldots{}
\item
  and that.
\end{itemize}

\ldots{} to get: ToDo: This \ldots{} and that.

\begin{itemize}
\tightlist
\item
  Notes Type:

  Note: take special care to \ldots{}
\end{itemize}

\ldots{} to get: Note: take special care to \ldots{}

\begin{itemize}
\tightlist
\item
  Linking text to Wikipedia \ldots{} using the
  \{\{WP\textbar{}\ldots{}\}\} template. If the linked text is the same
  as the Wikipedia page titke, simply type it: \{\{WP\textbar{}Mutual
  information\}\} Mutual information
\end{itemize}

If the linked text is different, use the ``\textbar{}'' pipe character
to separate page-name and text: \{\{WP\textbar{}Mutual
information\textbar{}`''WP article on Mutual Information'''\}\} WP
article on Mutual Information

\begin{itemize}
\tightlist
\item
  Collapsible elements See: Manual: Collapsible elements

  Visible text \ldots{}
\item
  Collapsed text 1
\item
  Collapsed text 2
\item
  Collapsed text 3
\end{itemize}

Example: {[}Expand for poem{]} The Road Not Taken Robert Frost

Sample reference section ↑ Steipe (2004) Consensus-based engineering of
protein stability: from intrabodies to thermostable enzymes. Meth
Enzymol 388:176-86. (pmid: 15289071) {[} PubMed {]}{[} DOI {]}

\subsection{\texorpdfstring{The ``User space'' and
subpages}{The User space and subpages}}\label{the-user-space-and-subpages}

The User: namespace on the Student Wiki is especially important.
Namespaces allow us to distinguish pages that share the same logical
name. Every student will create a journal page, but of course there can
be only one {[}{[}Journal{]}{]} page on the Wiki. Therefore each of
these pages needs a distinct name. The obvious solution is to keep them
in the User: namespace, and create them as subpages of everyone's User
page. The page name of your user page is {[}{[}User:{]}{]}; subpages are
created with a backslash, and therefore your Course Journal page should
be {[}{[}User:/Journal{]}{]}. if you take more than one course, you can
separate the journals like {[}{[}User:/BCH441-Journal{]}{]},
{[}{[}User:/BCB410-Journal{]}{]}, etc. Please do not create pages in the
``Main space'' of the Student Wiki! Do not omit the User:/ part of the
page name.

\subsection{Copyright}\label{copyright}

Over the last decades, in bioinformatics and many other fields of
science, the paradigm under which we create value has profoundly
changed. While we previously considered restrictions on the use of our
insights important, tried to keep knowledge under control, and thought
in terms of intellectual property, the modern paradigm is mindshare. We
strive to make our work maximally useful to others, and to document how
we are creating this utility. This does not mean that we are simply
putting everything into the public domain: yes, people should use our
ideas, but we must receive credit - as a currency for grant and
scholarship applications and the like, to enable our future work. The
right tool for this is copyright. Everything we write and create
automatically falls under our copyright, there is no special copyright
tag required. To have our material reused, we can either relinquish our
copyright or grant a license to reuse. Material that is created in
coursework will ideally be useful elsewhere, but it is only useful if
its use is permitted and regulated. Wikis are tools for collaboration,
and Wikipedia generally applies a site-wise license to all material. In
our work we take a similar approach, but we apply licenses more
specifically{[}1{]}. All material submitted for credit, including code,
documentation, essays, manuals, images, lab journal entries, insights!
pages etc. must be licensed with an appropriate open-source license.
This is a strict requirement for the course. For code this is the MIT
software license, for everything else this is the Creative Commons
Attribution 4.0 International License. The MIT license for code
guarantees that there are no restrictions on re-use other than fair and
visible attribution of the authors' work. The CC license guarantees
proper attribution of authorship but allows free use otherwise.
Together, these licenses allow the material to be used, refactored,
updated and republished and thus (hopefully) give it a fertile future
life. In order to keep copyright and licenses consistent throughout the
site, we use a template tag - simply insert it at the bottom of a page:
Entering the template code \ldots{} \{\{CC-BY\}\} creates the copyright
message \ldots{}

This copyrighted material is licensed under a Creative Commons
Attribution 4.0 International License. Follow the link to learn more.

\subsection{Task:}\label{task-1}

\begin{itemize}
\tightlist
\item
  Practice basic editing syntax by putting contents on your User Page:

  \begin{itemize}
  \tightlist
  \item
    enter your name,
  \item
    your major(s), specialist program, year of study - or a link to your
    lab and your thesis theme if you are a graduate student;
  \item
    enter your email address. I use this information a lot when I need
    to contact students, so make sure it is correct and current.
  \item
    Add a category tag to your User page for the course you are taking.
    All pages with this tag are accessible via the link in the sidebar.
    What should the category tag say? Good question \ldots{} go and find
    out.
  \item
    Add a copyright template to the bottom of your user page by putting
    a \{\{CC-BY\}\} tag on its bottom.
  \item
    Feel free to look at my User Page for code examples: clicking on the
    edit link will show you the source text. How do you find my User
    Page? Good question \ldots{}
  \item
    Create a subpage to your User Page; call it ``Journal''. Note: the
    link MUST be in your ``User space''. If you don't add the prefix
    User:yourname/\ldots{} before your page name, the new page will end
    up in the main ``namespace''. I'll then have to delete it. That's
    not good. Make sure you know what you are doing, for example by
    looking at the code on my User Page, asking someone who knows, or
    asking on the mailing list.
  \item
    Put some placeholder text on your journal page, you will fill it in
    when you work through the Journal unit.
  \item
    Similarly, create an ``insights!'' page on a subpage to your User
    Page and add some placeholder text. That will be expanded when you
    work through the insights! unit. Play around some more. Feel free to
    ask how to go about achieving a particular effect that you may have
    seen elsewhere.
  \end{itemize}
\end{itemize}

\subsection{Self-evaluation}\label{self-evaluation}

You should be familiar with the following: * How to Login to the Student
Wiki and access your user page; * viewing a page's history; * basic text
formatting and Wiki markup; * ``signing'' your name; * creating internal
and external links; * creating section headers on a page on multiple
levels; * reverting a changed page to an earlier version; * creating a
new page (as a subpage of an existing page); * the concept of namespaces
- especially the default (``main'') and User: namespace; * the concept
of categories and how to add a page to a category; * copyright on the
Student Wiki, and how to insert a license note.

\subsection{Notes}\label{notes}

`''Note''' that additional rules for collaboration in the context of
coursework derive from the rules for academic integrity and plagiarism.
If some text is not copyrighted, this does not mean you can use it
without reference and thus imply it is your own idea. That would be
plagiarism. Further reading, links and resources

If in doubt, ask! If anything about this learning unit is not clear to
you, do not proceed blindly but ask for clarification. Post your
question on the course mailing list: others are likely to have similar
problems. Or send an email to your instructor.

About \ldots{}

Author: Boris Steipe
\href{mailto:boris.steipe@utoronto.ca}{\nolinkurl{boris.steipe@utoronto.ca}}
Created: 2017-08-05 Modified: 2019-01-04 Version: 1.1 Version history:
1.1 Changed software license from GNU-GPL to MIT 1.0 Completed
outcomes/objectives. Added copyright. First live version. 0.2 First
contents imported from Help:editing. Added tasks. 0.1 First stub

\chapter{Your Course Journal}\label{journal}

(How to keep a course- or lab journal)

\section{Abstract:}\label{abstract-1}

Keeping a journal is an essential task in a laboratory. To practice
keeping a technical journal, you will document your activities as you
are working through the material of the course. A significant part of
your term grade will be given for this Course Journal. This unit
introduces components and best practice for lab- and course journals and
includes a wiki-source template to begin your own journal on the Student
Wiki.

\section{Objectives:}\label{objectives-1}

\begin{itemize}
\tightlist
\item
  Introducing components and best practice of lab- and course journals
\item
  Presenting sample wiki-text for Journal entries
\end{itemize}

\section{Outcomes:}\label{outcomes-1}

Upon concluding this unit you should be able to \ldots{} * Begin a
structured course journal on the Student Wiki using proper wiki text; *
Write your own journal entries, including media images and code as
required; * Cross-reference journal entries with links; * Link to
external sources and deliverables on internal pages as appropriate; *
Estimate the time you need for tasks, and develop a habit of improving
your time-management skills.

\section{Deliverables:}\label{deliverables-1}

\begin{enumerate}
\def\labelenumi{\arabic{enumi}.}
\tightlist
\item
  Your Journal: Your entire journal will be evaluated at the end of the
  course. Refer to the marking rubrics for details.
\item
  Insights: If you find something particularly noteworthy about this
  unit, make a note in your insights! page. \textbf{Caution:}
\item
  Your course journal is a deliverable of this course and it will be
  graded. Therefore all rules regarding plagiarism and other academic
  misconduct apply in full. In particular:

  \begin{itemize}
  \tightlist
  \item
    `''do not include any material from elsewhere without referencing
    it:''' We are operating a ``full disclosure'' policy in this course.
    Anything that you did not write yourself, on the spot, must be
    referenced. In particular you need to reference if you are copying
    your own material from other courses.;
  \item
    `''do not fabricate material that you are posting in your
    journal.''' Fabrication could include things like: modifying results
    produced by your code, describing work that you have not actually
    done, or claiming a time for the journal entry that is not the
    time/date on which it was actually written. All of these are
    academic offences.;
  \item
    `''Note:''' Only journal entries that were written concurrently with
    the activity they describe will be evaluated for credit.
  \item
    `''Note:''' All journal pages on the Student Wiki---like all other
    submitted material---must contain a \{\{CC-BY\}\} template.
  \end{itemize}
\end{enumerate}

\section{Prerequisites:}\label{prerequisites}

You need the following preparation before beginning this unit. If you
are not familiar with this material from courses you took previously,
you need to prepare yourself from other information sources: *
`''Inquiry'`': The scientific method; evidence based reasoning; how to
design, execute and document an experiment; Conjecture, hypothesis and
theory. *'`'Writing''': Basic essay and report writing skills. How to
format your submitted materials, how to quote, cite and avoid
plagiarism. * This unit builds on material covered in the following
prerequisite units: * \ref{wiki}

\begin{center}\rule{0.5\linewidth}{\linethickness}\end{center}

Work through this unit, then make your work with the ``Plagiarism'' Unit
the first entry of your Journal!

Computational research embraces the same best-practice principles as any
wet-lab experiment. We ensure our work is reproducible, we take great
care that our conclusions are supported by data, and we keep notes to
document our objectives, activities and how we arrived at our results.
Those notes are more than just a handy collection of information: they
need to become a robust, testable record of activities. Paper notes are
not very useful for bioinformatics work because they can't be
cross-referenced easily with computer files. Ideally, bioinformatics
journals will document results, and link to data files, code
repositories, Webpages and other resources. Thus a technical solution
needs to support incorporating or linking to results, data, code,
workflow scripts, documentation, and much more. In this course, we use
the open source Media Wiki software to support journal keeping{[}1{]}.
Keeping a record of your activities is a habit, and habits need to be
formed through practice. Is this going to be useful to you? I don't
know, but neither do you unless this habit has been given a credible
chance to form. Therefore we practice keeping journals in this course.
As a welcome side effect, this creates a record of activities for future
reference, and provide a basis for evaluation of your progress at the
end of the course. Keeping a journal will help you work with other
learning units or project components effectively, because this is all
integrated over the entire course, and later units often make use of
earlier results which you should have easily accessible. Remember: you
are writing a lab notebook---not a formal lab report: a point-form
record of your actual activities.{[}2{]} Write such documentation as
notes to your (future) self. Record everything that's necessary, but be
light and agile about your writing. Write your notes immediately, in
parallel with your actual activities, don't draft them elsewhere and
expect to enter and revise them later. Practice shows that delayed
processing of journal notes creates an unmanageable burden. Therefore
notes that are not written concurrently with the activity will not be
considered for credit in this course. This too is about habit forming.
But writing concurrently is so easy: since all of your computational
work is done with a computer, begin every work-session by opening an
editing window for its journal entry. Have the window open, and
immediately record everything of importance. The Wiki is online, so you
can even edit your journal from a library computer, and even (although
it's awkward) from your phone. Obviously, the first step is to create a
journal page in the User space of the Student Wiki - you have already
done this in the Wiki editing unit.

\section{Header}\label{header}

Write a header and give it a unique number. This is useful so you can
refer to the header number in later text. Obviously, you should
``hard-code'' the number and not use the Wiki's automatic section
numbering scheme, since the numbers should be stable over time, not
change when you add or delete a section{[}3{]}. It is useful to add any
new contents at the top of the page. Keeping the page in reverse
chronological order, prevents you having to scroll to the bottom of the
page every time you add new material. Note though, that the sections do
not actually have to be in strict chronological order, like we would
have them in a paper notebook. Typically you would number in a decimal
system - like 1, 1.1, 1.2, 2, 3 etc. - so you can easily accomodate
additions. It may be advantageous to give different subprojects their
own numbering space - by adding a prefix to the section number. This
depends on how related the projects are. Everything you keep on the same
page is easy to find with your browser's search function. But if search
results come from different projects, that may be inconvenient. To
decide what to put on the same page and what should go in different
subpages, imagine what material you would search for and what search
terms you might use{[}4{]}. Incidentally: the material in such a
notebook is ``permanent'', since earlier versions of pages are always
available via the history function. The Wiki never forgets. As well,
they are automatically time-stamped. And that's actually a step beyond
paper labnotes.

\section{Objective}\label{objective}

\begin{itemize}
\tightlist
\item
  State the objective.
\item
  In one brief sentence, restate what your activity is supposed to
  achieve.
\end{itemize}

\section{Estimate duration}\label{estimate-duration}

The learning units in this course require you to estimate beforehand how
long you will take, and to record how much time you actually took.
Record your initial estimate (work-hours), how many hours you took, and
how much time elapsed between start and end of your task. Make this a
habit in your future coursework as well as in your future labwork. You
will quickly note that you will become much better at time-management.
The sample journal template that is included below contains wikitext to
format a time estimate.

\section{What to document}\label{what-to-document}

\begin{itemize}
\tightlist
\item
  Document the procedure - Note what you have done, as concisely as
  possible but with sufficient detail. ``What is sufficient detail?''
  The answer is easy: detailed enough so that someone can reproduce what
  you have done. In practice that ``someone'' will often be you,
  yourself, in the future. I hope that you won't be constantly cursing
  your past-self because of omissions!
\item
  Document your results.

  \begin{itemize}
  \tightlist
  \item
    You can distinguish different types of results -
  \end{itemize}

  \begin{enumerate}
  \def\labelenumi{\arabic{enumi}.}
  \tightlist
  \item
    Static data does not change over time and it may be sufficient to
    note a reference to the result. For example, there is no need to
    copy a GenBank record into your documentation, it is sufficient to
    note the accession number, the refSeq ID, or the UniProt ID, or even
    better, to link to the relevant page on the external database
    server.
  \item
    Variable data can change over time. For example the results of a
    BLAST search depend on the sequences in the database. A list of
    similar structures may change as new structures get solved and
    deposited in the PDB database. In principle you want to record such
    data, to be able to reproduce at a later time what your conclusions
    were based on. But be selective in what you record. For example you
    should not paste the entire set of results of a BLAST search into
    your document, but only those matches that were important for your
    conclusions. `''Indiscriminate pasting of irrelevant information
    will make your notes unusable.''' Incidentally, the technology to
    expand and collapse paragraphs that we demonstrated in the Wiki
    editing unit can be put to excellent use to record data but keep it
    out of sight when not needed.
  \item
    Analysis results - The results of sequence analyses, alignments etc.
    in general get recorded in your documentation. Again: be selective.
    Record what is important.
  \end{enumerate}
\end{itemize}

\section{Conclusions}\label{conclusions}

Note your conclusions. - An analysis is not complete unless you conclude
something from the results. * Are two sequences likely homologues, or
not? Just pasting the BLAST output is not enough. It's your call -
`''record it'''. * Does your protein contain a signal-sequence or does
it not? SignalP will give you a probability, but you must make the final
call. * Is a binding site conserved, or not? The programs can only point
out sections of similarity or dissimilarity. You are the one who
interprets these numbers in their biological context. The analysis
provides the data. In your conclusion you provide the interpretation of
what the data means in the context of your objective. Were you expecting
a signal-sequence but there isn't one? What could that mean? Sometimes
your task will explicitly include to elaborate on an analysis and
conclusion. But this does not mean that when analysis is not explicitly
mentioned, you can skip the interpretation. In general you can never
expect full marks if analysis and conclusions are missing.

\section{Outlook for the next tasks}\label{outlook-for-the-next-tasks}

What's the next step? Note it here. Also include a link to the logically
next entry - this way you can quickly hop through consecutive entries
for a theme.

\section{Cross references}\label{cross-references}

Add cross-references. Cross-references to other information are
supremely valuable as your documentation grows. It's easy to see how to
format a link to a section of your Wiki-page: just look at the link
under the Table of Contents at the top. But you can also place
``anchors'' for linking anywhere on an HTML page: just use the following
syntax. \textless{}span
id=``\{some-label\}''\textgreater{}\textless{}\textbackslash{}span\textgreater{}
for the anchor, and append \#\{some-label\} to the page URL.

\section{Media}\label{media}

\subsection{Images}\label{images}

\begin{itemize}
\tightlist
\item
  Use discretion when uploading images
\item
  Don't upload irrelevant images, don't upload copyrighted images, keep
  the size reasonable. Prepare your images well
\item
  Don't upload uncompressed screen dumps. Save images in a compressed
  file format on your own computer. Then use the Image Upload link in
  the left-hand menu to upload images. The Wiki will only accept .jpeg,
  .png, .gif, or .svg images.
\item
  Use the correct image types.
\item
  In principle, images can be stored uncompressed as .tiff or .bmp, or
  compressed as .gif or .jpg or .png. .gif is useful for images with
  large, monochrome areas and sharp, high-contrast edges because the LZW
  compression algorithm it uses works especially well on such data; .jpg
  (or .jpeg) is preferred for images with shades and halftones such as
  the structure views you should prepare for several assignments, JPEG
  has excellent application support and is the most versatile general
  purpose image file format currently in use; .tiff (or .tif) is
  preferred to archive master copies of images in a lossless fashion,
  use LZW compression for TIFF files if your system/application supports
  it; The .png format is an open source alternative for lossless,
  compressed images. .bmp is not preferred for really anything, it is
  bloated in its (default) uncompressed form and primarily used only
  because it is simple to code and ubiquitous on Windows computers.
  Accordingly we don't support it here.
\end{itemize}

\subsection{Image dimensions and
resolution}\label{image-dimensions-and-resolution}

Stereo images should have equivalent points displayed approximately 6cm
apart. It depends on your monitor how many pixels this corresponds to.
The dimensions of an image are stated in pixels (width x height). My
notebook screen has a native display resolution of 1440 x 900
pixels/23.5 x 21 cm. Therefore a 6cm separation on my notebook
corresponds to approximately 260 pixels. However on my desktop monitor,
260 pixels is 6.7 cm across. And on a high-resolution iPad display, at
227 ppi (pixels per inch), 260 pixels are just 2.9 cm across. If your
assigment or learning unit ask you to prepare stero images: adjust your
images so they are approximately at the right separation and are
approximately 500 to 600 pixels wide. Also, scale your molecules so they
fill the available window and - if you have depth cueing enabled - move
them close to the front clipping plane so the molecule is not just a dim
blob, lost in murky shadows. Considerations for print (manuscripts etc.)
are slightly different: for print output you can specify the output
resolution in dpi (dots per inch). A typical print resolution is about
300 dpi: 6 cm separation at 300dpi is about 700 pixels. Print images
should therefore be about three times as large in width and height as
screen images.

\subsection{Preparation of stereo
views}\label{preparation-of-stereo-views}

\begin{itemize}
\tightlist
\item
  When assignments or leartning units ask you to create images of
  molecules, always create stereo views.
\item
  Keep your images uncluttered and expressive
\item
  Scale the molecular model to fill the available space of your image
  well. Orient views so they illustrate a point you are trying to make.
  Emphasize residues that you are writing about with a contrasting
  colouring scheme. Add labels, where residue identities are not
  otherwise obvious. Turn off side-chains for residues that are not
  important. The more you practice these small details, the more
  efficient you will become in the use of your tools.
\end{itemize}

\subsection{Code}\label{code}

Always markup code using the GeSHi extension. This provides syntax
highlighting, which is very useful to read the code. You simply place
the code-block into opening- and closing ``source'' tags, and tell GeSHi
which language it should assume. For R-code this looks like: Code
\ldots{}

{[}Expand{]} Expand for GeSHi rendered R code example \ldots{} You can
also use GeSHi to markup plain text - (although you can achieve a
similar effect by simply beginning each line with a blank " ``). Lorem
ipsum dolor sit amet \ldots{}

Documents \ldots{}

The section below contains Wiki-markup code that you can copy and paste
for your course journal.

Wikitext Template

\subsection{Self-evaluation}\label{self-evaluation-1}

\subsection{Notes}\label{notes-1}

Here are some alternative applications -- but (!) disclaimer, I myself
don't use any of these (yet).. 1. Evernote - a web hosted, automatically
syncing e-notebook. 1. Nevernote - the Open Source alternative to
Evernote. 1. Google Keep - if you have a Gmail account, you can simply
log in here. Grid-based. Seems a bit awkward for longer notes. But of
course you can also use Google Docs. 1. Microsoft OneNote - this sounds
interesting and if any one is using this, I'd like to hear from you.
Syncing across platforms, being able to format contents and organize it
sounds great. 1. RStudio projects - for development-focussed work --
especially (but not exclusively) -- in R, an RStudio project may be the
right solution to keep your code, results, notes, manuscript drafts,
literature and other assets all in one place. The great benefit is that
it can all be under version control and it's super easy to share
everything with colleagues on a team through GitHub Technically, GitHub
documents are all publicly accessible if they are stored in repositories
of free accounts - but you can commit binary files, so you can simply
keep sensitive material in password-protected .zip files or otherwise
encrypted.. The only downside that I can think of is that it's not
possible to cross-reference and link to material. 1. I have come across
``journal entries'' that consist only of copy/pasted learning unit
objectives\ldots{} 1. If the Wiki automatically displays section numbers
in its Table of Contents, you can turn that off in the preferences. 1.
Media Wiki also has its own search functions that search for material
everywhere on the Wiki, but this is likely not useful on the Student
Wiki where many users may be writing about similar things.

\subsection{Further reading, links and
resources}\label{further-reading-links-and-resources}

If in doubt, ask! If anything about this learning unit is not clear to
you, do not proceed blindly but ask for clarification. Post your
question on the course mailing list: others are likely to have similar
problems. Or send an email to your instructor.

About \ldots{}

Author: Boris Steipe
\href{mailto:boris.steipe@utoronto.ca}{\nolinkurl{boris.steipe@utoronto.ca}}
Created: 2017-08-05 Modified: 2019-01-05 Version: 1.3 Version history:
1.3 Emphasize habit forming and cuncurrent editing. Note on license. 1.2
Make time tags mandatory; warn against fabrication. 1.1 Add GeSHi
example 1.0 First live version 0.1 First stub

\chapter{Prerequisites}\label{prerequisites-1}

This is a \emph{sample} book written in \textbf{Markdown}. You can use
anything that Pandoc's Markdown supports, e.g., a math equation
\(a^2 + b^2 = c^2\).

The \textbf{bookdown} package can be installed from CRAN or Github:

\begin{Shaded}
\begin{Highlighting}[]
\KeywordTok{install.packages}\NormalTok{(}\StringTok{"bookdown"}\NormalTok{)}
\CommentTok{# or the development version}
\CommentTok{# devtools::install_github("rstudio/bookdown")}
\end{Highlighting}
\end{Shaded}

Remember each Rmd file contains one and only one chapter, and a chapter
is defined by the first-level heading \texttt{\#}.

To compile this example to PDF, you need XeLaTeX. You are recommended to
install TinyTeX (which includes XeLaTeX):
\url{https://yihui.name/tinytex/}.

\section{Attributions:}\label{attributions}

Original content for this book is from
\href{http://steipe.biochemistry.utoronto.ca/abc/index.php/Computational_Systems_Biology_Main_Page}{Boris
Steipe BCB420 wiki resources} licensed under
\includegraphics{images/cc_icon.png}
\href{https://creativecommons.org/licenses/by/4.0/}{CC BY 4.0}.

Icons are from the
\href{https://www.iconfinder.com/iconsets/very-basic-android-l-lollipop}{``Very
Basic. Android L Lollipop'' set by Ivan Boyko} licensed under
\href{https://creativecommons.org/licenses/by/3.0/}{CC BY 3.0}.

\bibliography{book.bib,packages.bib}


\end{document}
