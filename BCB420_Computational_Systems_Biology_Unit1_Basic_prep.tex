\documentclass[]{book}
\usepackage{lmodern}
\usepackage{amssymb,amsmath}
\usepackage{ifxetex,ifluatex}
\usepackage{fixltx2e} % provides \textsubscript
\ifnum 0\ifxetex 1\fi\ifluatex 1\fi=0 % if pdftex
  \usepackage[T1]{fontenc}
  \usepackage[utf8]{inputenc}
\else % if luatex or xelatex
  \ifxetex
    \usepackage{mathspec}
  \else
    \usepackage{fontspec}
  \fi
  \defaultfontfeatures{Ligatures=TeX,Scale=MatchLowercase}
\fi
% use upquote if available, for straight quotes in verbatim environments
\IfFileExists{upquote.sty}{\usepackage{upquote}}{}
% use microtype if available
\IfFileExists{microtype.sty}{%
\usepackage{microtype}
\UseMicrotypeSet[protrusion]{basicmath} % disable protrusion for tt fonts
}{}
\usepackage{hyperref}
\hypersetup{unicode=true,
            pdftitle={BCB420 - Computational System Biology},
            pdfauthor={Main author: Boris Steipe; Modified: Ruth Isserlin},
            pdfborder={0 0 0},
            breaklinks=true}
\urlstyle{same}  % don't use monospace font for urls
\usepackage{color}
\usepackage{fancyvrb}
\newcommand{\VerbBar}{|}
\newcommand{\VERB}{\Verb[commandchars=\\\{\}]}
\DefineVerbatimEnvironment{Highlighting}{Verbatim}{commandchars=\\\{\}}
% Add ',fontsize=\small' for more characters per line
\usepackage{framed}
\definecolor{shadecolor}{RGB}{248,248,248}
\newenvironment{Shaded}{\begin{snugshade}}{\end{snugshade}}
\newcommand{\KeywordTok}[1]{\textcolor[rgb]{0.13,0.29,0.53}{\textbf{#1}}}
\newcommand{\DataTypeTok}[1]{\textcolor[rgb]{0.13,0.29,0.53}{#1}}
\newcommand{\DecValTok}[1]{\textcolor[rgb]{0.00,0.00,0.81}{#1}}
\newcommand{\BaseNTok}[1]{\textcolor[rgb]{0.00,0.00,0.81}{#1}}
\newcommand{\FloatTok}[1]{\textcolor[rgb]{0.00,0.00,0.81}{#1}}
\newcommand{\ConstantTok}[1]{\textcolor[rgb]{0.00,0.00,0.00}{#1}}
\newcommand{\CharTok}[1]{\textcolor[rgb]{0.31,0.60,0.02}{#1}}
\newcommand{\SpecialCharTok}[1]{\textcolor[rgb]{0.00,0.00,0.00}{#1}}
\newcommand{\StringTok}[1]{\textcolor[rgb]{0.31,0.60,0.02}{#1}}
\newcommand{\VerbatimStringTok}[1]{\textcolor[rgb]{0.31,0.60,0.02}{#1}}
\newcommand{\SpecialStringTok}[1]{\textcolor[rgb]{0.31,0.60,0.02}{#1}}
\newcommand{\ImportTok}[1]{#1}
\newcommand{\CommentTok}[1]{\textcolor[rgb]{0.56,0.35,0.01}{\textit{#1}}}
\newcommand{\DocumentationTok}[1]{\textcolor[rgb]{0.56,0.35,0.01}{\textbf{\textit{#1}}}}
\newcommand{\AnnotationTok}[1]{\textcolor[rgb]{0.56,0.35,0.01}{\textbf{\textit{#1}}}}
\newcommand{\CommentVarTok}[1]{\textcolor[rgb]{0.56,0.35,0.01}{\textbf{\textit{#1}}}}
\newcommand{\OtherTok}[1]{\textcolor[rgb]{0.56,0.35,0.01}{#1}}
\newcommand{\FunctionTok}[1]{\textcolor[rgb]{0.00,0.00,0.00}{#1}}
\newcommand{\VariableTok}[1]{\textcolor[rgb]{0.00,0.00,0.00}{#1}}
\newcommand{\ControlFlowTok}[1]{\textcolor[rgb]{0.13,0.29,0.53}{\textbf{#1}}}
\newcommand{\OperatorTok}[1]{\textcolor[rgb]{0.81,0.36,0.00}{\textbf{#1}}}
\newcommand{\BuiltInTok}[1]{#1}
\newcommand{\ExtensionTok}[1]{#1}
\newcommand{\PreprocessorTok}[1]{\textcolor[rgb]{0.56,0.35,0.01}{\textit{#1}}}
\newcommand{\AttributeTok}[1]{\textcolor[rgb]{0.77,0.63,0.00}{#1}}
\newcommand{\RegionMarkerTok}[1]{#1}
\newcommand{\InformationTok}[1]{\textcolor[rgb]{0.56,0.35,0.01}{\textbf{\textit{#1}}}}
\newcommand{\WarningTok}[1]{\textcolor[rgb]{0.56,0.35,0.01}{\textbf{\textit{#1}}}}
\newcommand{\AlertTok}[1]{\textcolor[rgb]{0.94,0.16,0.16}{#1}}
\newcommand{\ErrorTok}[1]{\textcolor[rgb]{0.64,0.00,0.00}{\textbf{#1}}}
\newcommand{\NormalTok}[1]{#1}
\usepackage{longtable,booktabs}
\usepackage{graphicx}
% grffile has become a legacy package: https://ctan.org/pkg/grffile
\IfFileExists{grffile.sty}{%
\usepackage{grffile}
}{}
\makeatletter
\def\maxwidth{\ifdim\Gin@nat@width>\linewidth\linewidth\else\Gin@nat@width\fi}
\def\maxheight{\ifdim\Gin@nat@height>\textheight\textheight\else\Gin@nat@height\fi}
\makeatother
% Scale images if necessary, so that they will not overflow the page
% margins by default, and it is still possible to overwrite the defaults
% using explicit options in \includegraphics[width, height, ...]{}
\setkeys{Gin}{width=\maxwidth,height=\maxheight,keepaspectratio}
\IfFileExists{parskip.sty}{%
\usepackage{parskip}
}{% else
\setlength{\parindent}{0pt}
\setlength{\parskip}{6pt plus 2pt minus 1pt}
}
\setlength{\emergencystretch}{3em}  % prevent overfull lines
\providecommand{\tightlist}{%
  \setlength{\itemsep}{0pt}\setlength{\parskip}{0pt}}
\setcounter{secnumdepth}{5}
% Redefines (sub)paragraphs to behave more like sections
\ifx\paragraph\undefined\else
\let\oldparagraph\paragraph
\renewcommand{\paragraph}[1]{\oldparagraph{#1}\mbox{}}
\fi
\ifx\subparagraph\undefined\else
\let\oldsubparagraph\subparagraph
\renewcommand{\subparagraph}[1]{\oldsubparagraph{#1}\mbox{}}
\fi

%%% Use protect on footnotes to avoid problems with footnotes in titles
\let\rmarkdownfootnote\footnote%
\def\footnote{\protect\rmarkdownfootnote}

%%% Change title format to be more compact
\usepackage{titling}

% Create subtitle command for use in maketitle
\providecommand{\subtitle}[1]{
  \posttitle{
    \begin{center}\large#1\end{center}
    }
}

\setlength{\droptitle}{-2em}

  \title{BCB420 - Computational System Biology}
    \pretitle{\vspace{\droptitle}\centering\huge}
  \posttitle{\par}
    \author{Main author: Boris Steipe; Modified: Ruth Isserlin}
    \preauthor{\centering\large\emph}
  \postauthor{\par}
      \predate{\centering\large\emph}
  \postdate{\par}
    \date{last modified 2019-12-23}


\let\BeginKnitrBlock\begin \let\EndKnitrBlock\end
\begin{document}
\maketitle

{
\setcounter{tocdepth}{1}
\tableofcontents
}
\chapter{About}\label{about}

Original content for this book was created by
\href{http://biochemistry.utoronto.ca/person/boris-steipe/}{Boris
Steipe} from
\href{http://steipe.biochemistry.utoronto.ca/abc/index.php/Computational_Systems_Biology_Main_Page}{Boris
Steipe BCB420 wiki resources} licensed under
\includegraphics{images/cc_icon.png}
\href{https://creativecommons.org/licenses/by/4.0/}{CC BY 4.0}.

\section{Attributions:}\label{attributions}

This book was created using The \textbf{bookdown} package and can be
installed from CRAN or Github:

\begin{Shaded}
\begin{Highlighting}[]
\KeywordTok{install.packages}\NormalTok{(}\StringTok{"bookdown"}\NormalTok{)}
\CommentTok{# or the development version}
\CommentTok{# devtools::install_github("rstudio/bookdown")}
\end{Highlighting}
\end{Shaded}

Icons are from the
\href{https://www.iconfinder.com/iconsets/very-basic-android-l-lollipop}{``Very
Basic. Android L Lollipop'' set by Ivan Boyko} licensed under
\href{https://creativecommons.org/licenses/by/3.0/}{CC BY 3.0}.

\hypertarget{wiki}{\chapter{Wiki - Editing}\label{wiki}}

(Wiki editing, namespaces; user page setup; copyright, a Course Journal
stub page, and an insights! stub page)

\section{Overview}\label{overview}

\subsection{Abstract:}\label{abstract}

This will likely be the first learning unit you work with, since your
Course Journal will be kept on a Wiki, as well as all other
deliverables. This unit includes an introduction to authoring Wikitext
and the structure of Wikis, in particular how different pages live in
separate ``Namespaces''. The unit also covers the standard markup
conventions - ``Wikitext markup'' - the same conventions that are used
on Wikipedia - as well as some extensions that are specific to our
Course- and Student Wiki. We also discuss page categories that help keep
a Wiki organized, licensing under a Creative Commons Attribution
license, and how to add licenses and other page components through
template codes.

\subsection{Objectives:}\label{objectives}

\begin{itemize}
\tightlist
\item
  Provide an introduction to Wiki principles and Wikitext markup.
\item
  Create first pages on your own on the Student Wiki.
\item
  Learn about copyright, why we use Creative Commons licenses for the
  Student Wiki and how to place a license tag.
\end{itemize}

\subsection{Outcomes:}\label{outcomes}

\begin{itemize}
\tightlist
\item
  You are competent with basic Wiki markup and the extensions on this
  Wiki.
\item
  You can create pages and add them to categories while taking care to
  create them into your own user space.
\item
  You have created your own user page on the Student Wiki and added
  contents.
\item
  You have created page stubs for a Course Journal and an insights!
  page.
\end{itemize}

\subsection{Deliverables:}\label{deliverables}

\begin{itemize}
\tightlist
\item
  Specified as ``Tasks'': There are no general deliverables for this
  unit; specific deliverables are described in the ``Task'' sections.
\end{itemize}

\hypertarget{wiki}{\section{Wiki}\label{wiki}}

Collaboration is a common theme for modern lab work and a Wiki is a
great way to share and seamlessly update information in groups - or just
for yourself. Probably the most sophisticated Wiki software is
MediaWiki, a set of PHP scripts that is under continuous development by
the Wikimedia foundation; it is the same software that runs Wikipedia.
This is open source, free software that is easy to install, is well
documented and requires very little resources other than a machine that
runs a MySQL database server and an Apache Webserver. Numerous
extensions exist (and extensions are not hard to write); they enhance
the already rich functionality. But let's start with small steps. You
should by now have a user account on the Student Wiki, and I have
configured that Wiki so that:

\begin{itemize}
\tightlist
\item
  only logged in users can view the pages; but \ldots{}
\item
  all logged in users can create and edit (most) pages at will.
\item
  This means you could edit pages that don't ``belong'' to you. Respect
  the ``House Rules'' and don't edit other's things without permission,
  even if you can think of a particularly witty comment or hilarious
  prank. If you want to comment on a page: every page has an associated
  ``Discussion'' page that you can freely edit. Remember to ``sign your
  name'' to discussion entries.
\end{itemize}

\begin{rmd-task}
\begin{enumerate}
\def\labelenumi{\arabic{enumi}.}
\tightlist
\item
  Access the Student Wiki;
\item
  log in and navigate to your user page. (Your user page is linked to
  your name that appears at the top of every Wikipage once you are
  logged in.)
\item
  Create / edit the page, try out and experiment with the Wikitext
  syntax that this unit covers as you read about the different elements.
\item
  Work through the contents below.
\end{enumerate}
\end{rmd-task}

For more extensive formatting help
see:\url{http://meta.wikimedia.org/wiki/Help:Editing}

For Math markup see: \url{http://meta.wikimedia.org/wiki/Help:Formula}

\section{The Wiki concept}\label{the-wiki-concept}

Wiki sites are collections of Web pages that allow you to view, edit and
create pages from your browser, there is no need for special technology
and basic editing is simple and intuitive with ``Wikitext markup''.

The basic workflow of Wikis is straightforward:

\begin{itemize}
\tightlist
\item
  Register an account and browse the site.
\item
  Whenever you find something that you can improve, edit it.
\item
  Whenever you find something that you would like to comment on, click
  on the ``discussion'' tab and share your views.
\item
  If you are interested in what becomes of your edits or the discussion,
  click on the ``watch'' tab, and the page will be added to a list of
  bookmarks to pages you are ``watching''. (You can even generate an RSS
  feed for recent changes or new pages).
\item
  No e-mail, no obligations. Do what you like, when you like, what you
  can.
\item
  Editing on the Course Wiki is only enabled for instructors. However
  you can freely edit all pages on the Student Wiki, once you have an
  account.
\end{itemize}

\section{Editing basics}\label{editing-basics}

\subsection{Start editing}\label{start-editing}

To start editing a Wiki page, click on the ``Edit'' link at one of its
edges. This will bring you to the edit page: a page with a text box
containing the wikitext: the editable source code from which the server
produces the webpage.

\subsection{Preview before saving}\label{preview-before-saving}

When you have finished, press Show preview to see how your changes will
look. Repeat the edit/preview process until you are satisfied, then
click Save and your changes will be immediately applied to the article
and accessible on the Web. They will also be stored in the main database
for as long as the Wiki exists. Thus it is always possible to get back
earlier versions of pages - back to the very first edit.

\subsection{Basic text formatting}\label{basic-text-formatting}

Here are some examples of the markup of Wikitext. It is not the same as
HTML markup, however some HTML markup will work. In particular, the Wiki
applies styles through CSS technology (Cascading Style Sheets) and thus
HTML tags can be used to apply consistent styles to individual page
elements. Javascript won't work.

\begin{longtable}[]{@{}ll@{}}
\toprule
\begin{minipage}[b]{0.45\columnwidth}\raggedright\strut
What you type\strut
\end{minipage} & \begin{minipage}[b]{0.50\columnwidth}\raggedright\strut
What it looks like\strut
\end{minipage}\tabularnewline
\midrule
\endhead
\begin{minipage}[t]{0.45\columnwidth}\raggedright\strut
You can emphasize text by putting two apostrophes on each side. Three
apostrophes will emphasize it strongly. Five apostrophes is even
stronger.\strut
\end{minipage} & \begin{minipage}[t]{0.50\columnwidth}\raggedright\strut
You can \textbf{emphasize text} by putting two apostrophes on each side.
Three apostrophes will emphasize it \textbf{strongly}. Five apostrophes
is \textbf{even stronger}.\strut
\end{minipage}\tabularnewline
\begin{minipage}[t]{0.45\columnwidth}\raggedright\strut
A single newline has no effect on the layout.But an empty line starts a
new paragraph.\strut
\end{minipage} & \begin{minipage}[t]{0.50\columnwidth}\raggedright\strut
A single newline has no effect on the layout. But an empty line starts a
new paragraph.\strut
\end{minipage}\tabularnewline
\begin{minipage}[t]{0.45\columnwidth}\raggedright\strut
You can break lines without starting a new paragraph using a
\textless{}br\textgreater{} tag.\strut
\end{minipage} & \begin{minipage}[t]{0.50\columnwidth}\raggedright\strut
You can break lines without starting a new paragraph.\strut
\end{minipage}\tabularnewline
\bottomrule
\end{longtable}

(This may not be very useful beyond the types of examples we show here,
but it is a frequent source of confusion, when you find your text marked
up this way by accident) You can format text in a monospace font with a
dashed box around it either by marking it with the HTML
\textless{}pre\textgreater{} tag, or by putting a blank space at the
beginning of a line. Example:

\subsection{Other special characters at the beginning of a line
include:}\label{other-special-characters-at-the-beginning-of-a-line-include}

bulleted list

\begin{itemize}
\tightlist
\item
  bulleted list
\end{itemize}

numbered list

\begin{enumerate}
\def\labelenumi{\arabic{enumi}.}
\tightlist
\item
  numbered list
\end{enumerate}

term

; term

\begin{description}
\item[definition]
and definition
\end{description}

\subsection{Other special characters at the beginning of a line
include:}\label{other-special-characters-at-the-beginning-of-a-line-include-1}

You should ``sign'' your comments on discussion pages: Three tildes
gives your user name - Boris (talk) Four tildes: user name plus
date/time - Boris (talk) 22:18, 27 December 2012 (EST) Five tildes:
date/time alone - 22:18, 27 December 2012 (EST) You should ``sign'' your
comments on discusion pages: : Three tildes gives your user name -
\textsubscript{\textasciitilde{}} : Four tildes: user name plus
date/time - \textasciitilde{}\textsubscript{\textasciitilde{}} : Five
tildes: date/time alone -
\textsubscript{\textasciitilde{}}\textasciitilde{}\textasciitilde{}

\begin{itemize}
\tightlist
\item
  Use normal HTML character codes for special characters, or use
  Unicode. For example: \textgreater{} \textless{} \& ° Å Ä ü →
\item
  Use normal HTML character codes for special characters, or use
  Unicode. For example: \textgreater{} \textless{} \& ° Å Ä ü →
\end{itemize}

\subsection{You can use HTML tags, too, if you want. Some useful ways to
use
HTML:}\label{you-can-use-html-tags-too-if-you-want.-some-useful-ways-to-use-html}

\begin{longtable}[]{@{}ll@{}}
\toprule
\begin{minipage}[b]{0.45\columnwidth}\raggedright\strut
What you type\strut
\end{minipage} & \begin{minipage}[b]{0.50\columnwidth}\raggedright\strut
What it looks like\strut
\end{minipage}\tabularnewline
\midrule
\endhead
\begin{minipage}[t]{0.45\columnwidth}\raggedright\strut
Put text in a typewriter font (with \textless{}tt\textgreater{}). The
same font is generally used for computer code (with
\textless{}code\textgreater{}).\strut
\end{minipage} & \begin{minipage}[t]{0.50\columnwidth}\raggedright\strut
Put text in a typewriter font.The same font is generally used for
computer code.\strut
\end{minipage}\tabularnewline
\begin{minipage}[t]{0.45\columnwidth}\raggedright\strut
Strike out (with \textless{}string\textgreater{}) or underline
text(\textless{}u\textgreater{}), or write it in small caps(with
\textless{}span style=``font-variant:small-caps''\textgreater{}.\strut
\end{minipage} & \begin{minipage}[t]{0.50\columnwidth}\raggedright\strut
Strike out or underline text, or write it \textsc{ in small caps}.\strut
\end{minipage}\tabularnewline
\begin{minipage}[t]{0.45\columnwidth}\raggedright\strut
Superscripts (with \textless{}sup\textgreater{}) and subscripts(with
\textless{}sub\textgreater{}) : x2, x2\strut
\end{minipage} & \begin{minipage}[t]{0.50\columnwidth}\raggedright\strut
x2, x2\strut
\end{minipage}\tabularnewline
\begin{minipage}[t]{0.45\columnwidth}\raggedright\strut
Invisible comments that only appear while editing the page.\strut
\end{minipage} & \begin{minipage}[t]{0.50\columnwidth}\raggedright\strut
\strut
\end{minipage}\tabularnewline
\bottomrule
\end{longtable}

For a list of HTML tags that are allowed, see HTML in wikitext. I tend
to use Wiki-markup when I'm in a hurry, but use the HTML tag whenever I
can't remember a Wiki-tag. It really doesn't make a difference. However:
I never use Wiki-table markup. I find it less intuitive than HTML
markup, more difficult to debug, and there's really no point in
remembering both types of markup given that one really needs to be
comfortable with HTML tables anyway.

\subsection{Links}\label{links}

You will often want to make clickable links to other pages. What it
looks like What you type Here's a link to a page named Sandbox. You can
even say Sandboxes and the link will show up right. You can put
formatting around a link. Example: Sandbox. Here's a link to a page
named {[}{[}Sandbox{]}{]}. You can even say {[}{[}Sandbox{]}{]}es and
the link will show up right.

You can put formatting around a link. Example: `'{[}{[}Sandbox{]}{]}''.

You can link to an arbitrary piece of text with a piped link. Put the
link target first, then the pipe character ``\textbar{}'', then the link
text - as in this example. You can link an arbitrary piece of text with
a `'piped link''. Put the link target first, then the pipe character
``\textbar{}'', then the link text - as in {[}{[}Sandbox\textbar{} this
example{]}{]}.

You can make an external link to a Web page just by typing an URL,
e.g.~\url{http://igem.org} Or you can link arbitrary text: iGEM. (Note:
No ``\textbar{}'' for external links, URL and text are separated by a
blank, and only single square brackets!) Or you can generate a
footnote-like link:.\footnote{example footnote} You can make an external
link to a Web page just by typing an URL, e.g.~\url{http://igem.org}

Or you can link arbitrary text: {[}\url{http://igem.org} iGEM{]}. (Note:
No ``\textbar{}'' for `''external''' links, URL and text are separated
by a blank, and only single square brackets!)

Or you can generate a footnote-like link: {[}\url{http://igem.org}{]}.

\textbf{Note}: remember: internal links (using {[}{[}\ldots{}{]}{]} tags
to link to pages on this Wiki) are separated from linked text with a
pipe character. External links (using {[}\ldots{}{]} tags to link to
pages elsewhere on the Internet) are separated from linked text with a
space character.

\subsection{Special syntax}\label{special-syntax}

Two special syntax items need to be mentioned: ``templates'' and ``magic
words'': \#\#\#\# Templates Templates are pieces of Wikitext that are
substituted where a code that links to them has been placed into a page.
For example, if you enter \{\{Lorem\}\} on a page, the ``Lorem ipsum
dolor sit amet \ldots{}'' placeholder text is inserted in place of that
code. Wikis make extensive use of templates. \#\#\#\# Magic words some
reserved ``magic''-words are replaced with dynamically created contents
when the page is rendered. For example \textbf{TOC} forces placing a
Table Of Contents at the position of this token rather than its default
position, while \textbf{NOTOC} suppresses creation of a Table Of
Contents on a page.

\subsection{Creating a new page}\label{creating-a-new-page}

To create a new page simply insert a link to a Wiki page, which has a
page name that does not exist yet. The link will appear in red (except
if you inadvertently used the name of a page that already exists), and
the new page will be created when you click on the link. Page names can
be long and contain blank spaces. Internally, all blank spaces are
converted to underscore characters, but you can use the page name
without underscores in links; the Wiki software translates this for you.

\subsection{Namespaces}\label{namespaces}

The Wiki maintains some pages in special collections, in so called
``namespaces''. This is useful, because the behaviour of the software
can be customized for different namespaces: for example you may be
allowed to edit in the main- and the user- namespace, but not in the
MediaWiki: namespace, where pages are held that affect the gears and
wires of the Wiki. Page names without a prefix live in the main space.
Some commonly used prefixes are:

\begin{itemize}
\tightlist
\item
  User: - personal pages for user with an account on the Wiki;
\item
  Talk: - discussion pages for comments on pages, accessible via the
  ``Discussion'' tab;
\item
  Help: - this page for example;
\item
  Template: - pages with reusable text.
\item
  Special: - pages that implement special functionality (like login,
  user lists, or lists of recently changed pages);
\item
  Category: - an index of pages that have been given a common tag. This
  is a convenient way to access pages that are in some way related;
\end{itemize}

\subsection{Categories}\label{categories}

Once your page has been edited, you can associate it with one or more
categories. Add the appropriate category tag by typing
{[}{[}Category:BCH441\_2013{]}{]} or {[}{[}Category:BCB410\_2013{]}{]}.
The page is then automatically linked from a page that collects all
pages with that category tag. I would prefer that you do not create new
categories; ask me if you feel a need for it.

\subsection{Creating a new section or subsection on a
page}\label{creating-a-new-section-or-subsection-on-a-page}

To create a section or subsection, simply insert a section header into
an existing section. Header levels are defined by the number of ``=''
characters before and after the header text. Click on an edit link of
this page to see example code. Once a page has more than two headings,
the Wiki automatically creates a table of contents. You can adjust the
position of the table of contents by typing the ``magic word''
\_\_TOC\_\_somewhere on your page (Note: double underscore), you can
also suppress having a table of contents created with\textbf{NOTOC}.

\subsection{Edit conflicts}\label{edit-conflicts}

If someone else makes an edit while you are making yours, the result is
an edit conflict. Many conflicts can be automatically resolved by the
Wiki. If it can't be resolved, however, you will need to resolve it
yourself. The Wiki gives you two text boxes, where the top one is the
other person's edit and the bottom one is your edit. Merge your edits
into the top edit box, which is the only one that will be saved.

\subsection{Reverting pages to a previous
state}\label{reverting-pages-to-a-previous-state}

Sometimes a page needs to be reverted to a previous state. Access the
page through a link to the Recent Changes special page:
Special:Recentchanges. Find the page you need to revert, click on the
hist link, click on the version you need and verify that it is the
correct one. Then click on the edit tab at the top and Save page. A new
version of the page is then created with the old text. Note that this
does not actually overwrite anything - all edits are archived in the
database.

\section{\texorpdfstring{The ``User space'' and
subpages}{The User space and subpages}}\label{the-user-space-and-subpages}

The User: namespace on the Student Wiki is especially important.
Namespaces allow us to distinguish pages that share the same logical
name. Every student will create a journal page, but of course there can
be only one {[}{[}Journal{]}{]} page on the Wiki. Therefore each of
these pages needs a distinct name. The obvious solution is to keep them
in the User: namespace, and create them as subpages of everyone's User
page. The page name of your user page is {[}{[}User:{]}{]}; subpages are
created with a backslash, and therefore your Course Journal page should
be {[}{[}User:/Journal{]}{]}. if you take more than one course, you can
separate the journals like {[}{[}User:/BCH441-Journal{]}{]},
{[}{[}User:/BCB410-Journal{]}{]}, etc. Please do not create pages in the
``Main space'' of the Student Wiki! Do not omit the User:/ part of the
page name.

\section{Copyright}\label{copyright}

Over the last decades, in bioinformatics and many other fields of
science, the paradigm under which we create value has profoundly
changed. While we previously considered restrictions on the use of our
insights important, tried to keep knowledge under control, and thought
in terms of intellectual property, the modern paradigm is mindshare. We
strive to make our work maximally useful to others, and to document how
we are creating this utility. This does not mean that we are simply
putting everything into the public domain: yes, people should use our
ideas, but we must receive credit - as a currency for grant and
scholarship applications and the like, to enable our future work. The
right tool for this is copyright. Everything we write and create
automatically falls under our copyright, there is no special copyright
tag required. To have our material reused, we can either relinquish our
copyright or grant a license to reuse. Material that is created in
coursework will ideally be useful elsewhere, but it is only useful if
its use is permitted and regulated. Wikis are tools for collaboration,
and Wikipedia generally applies a site-wise license to all material. In
our work we take a similar approach, but we apply licenses more
specifically\footnote{\textbf{Note}:that additional rules for
  collaboration in the context of coursework derive from the rules for
  academic integrity and plagiarism. If some text is not copyrighted,
  this does not mean you can use it without reference and thus imply it
  is your own idea. That would be plagiarism.}. All material submitted
for credit, including code, documentation, essays, manuals, images, lab
journal entries, insights! pages etc. must be licensed with an
appropriate open-source license. This is a strict requirement for the
course. For code this is the MIT software license, for everything else
this is the Creative Commons Attribution 4.0 International License. The
MIT license for code guarantees that there are no restrictions on re-use
other than fair and visible attribution of the authors' work. The CC
license guarantees proper attribution of authorship but allows free use
otherwise. Together, these licenses allow the material to be used,
refactored, updated and republished and thus (hopefully) give it a
fertile future life. In order to keep copyright and licenses consistent
throughout the site, we use a template tag - simply insert it at the
bottom of a page: Entering the template code \ldots{} \{\{CC-BY\}\}
creates the copyright message \ldots{}

This copyrighted material is licensed under a Creative Commons
Attribution 4.0 International License. Follow the link to learn more.

\section{Task}\label{task}

\begin{rmd-task}
\begin{itemize}
\tightlist
\item
  Practice basic editing syntax by putting contents on your User Page:

  \begin{itemize}
  \tightlist
  \item
    enter your name,
  \item
    your major(s), specialist program, year of study - or a link to your
    lab and your thesis theme if you are a graduate student;
  \item
    enter your email address. I use this information a lot when I need
    to contact students, so make sure it is correct and current.
  \item
    Add a category tag to your User page for the course you are taking.
    All pages with this tag are accessible via the link in the sidebar.
    What should the category tag say? Good question \ldots{} go and find
    out.
  \item
    Add a copyright template to the bottom of your user page by putting
    a \{\{CC-BY\}\} tag on its bottom.
  \item
    Feel free to look at my User Page for code examples: clicking on the
    edit link will show you the source text. How do you find my User
    Page? Good question \ldots{}
  \item
    Create a subpage to your User Page; call it ``Journal''. Note: the
    link MUST be in your ``User space''. If you don't add the prefix
    User:yourname/\ldots{} before your page name, the new page will end
    up in the main ``namespace''. I'll then have to delete it. That's
    not good. Make sure you know what you are doing, for example by
    looking at the code on my User Page, asking someone who knows, or
    asking on the mailing list.
  \item
    Put some placeholder text on your journal page, you will fill it in
    when you work through the Journal unit.
  \item
    Similarly, create an ``insights!'' page on a subpage to your User
    Page and add some placeholder text. That will be expanded when you
    work through the insights! unit.
  \item
    Play around some more. Feel free to ask how to go about achieving a
    particular effect that you may have seen elsewhere.
  \end{itemize}
\end{itemize}
\end{rmd-task}

\section{Self-evaluation}\label{self-evaluation}

You should be familiar with the following: * How to Login to the Student
Wiki and access your user page; * viewing a page's history; * basic text
formatting and Wiki markup; * ``signing'' your name; * creating internal
and external links; * creating section headers on a page on multiple
levels; * reverting a changed page to an earlier version; * creating a
new page (as a subpage of an existing page); * the concept of namespaces
- especially the default (``main'') and User: namespace; * the concept
of categories and how to add a page to a category; * copyright on the
Student Wiki, and how to insert a license note.

\section{Further reading, links and
resources}\label{further-reading-links-and-resources}

\textbf{If in doubt, ask!} If anything about this learning unit is not
clear to you, do not proceed blindly but ask for clarification. Post
your question on the course mailing list: others are likely to have
similar problems. Or send an email to your instructor.

\BeginKnitrBlock{rmd-original-history}
\textbf{Author}: Boris Steipe
\href{mailto:boris.steipe@utoronto.ca}{\nolinkurl{boris.steipe@utoronto.ca}}
\textbf{Created}: 2017-08-05 \textbf{Modified}: 2019-01-04 Version: 1.1
\textbf{Version history}: 1.1 Changed software license from GNU-GPL to
MIT 1.0 Completed outcomes/objectives. Added copyright. First live
version. 0.2 First contents imported from Help:editing. Added tasks. 0.1
First stub
\EndKnitrBlock{rmd-original-history}

\subsection{Updated Revision history}\label{updated-revision-history}

\begin{tabular}{l|l|l|l}
\hline
Revision & Author & Date & Message\\
\hline
f56a24c & Ruth Isserlin & 2019-12-23 & Added new git info to each fileAdded new git info to each file (in addition to the original version history copied over from Boris's wiki).\\
\hline
8950904 & Ruth Isserlin & 2019-12-22 & Initial check in of converted wiki pages from Boris Steipe's bcb420 course material pagewiki pages were converted to bookdown and formatted to the bookdown format\\
\hline
\end{tabular}

\subsection{Footnotes:}\label{footnotes}

\hypertarget{journal}{\chapter{Your Course Journal}\label{journal}}

(How to keep a course- or lab journal)

\section{Overview}\label{overview-1}

\subsection{Abstract:}\label{abstract-1}

Keeping a journal is an essential task in a laboratory. To practice
keeping a technical journal, you will document your activities as you
are working through the material of the course. A significant part of
your term grade will be given for this Course Journal. This unit
introduces components and best practice for lab- and course journals and
includes a wiki-source template to begin your own journal on the Student
Wiki.

\subsection{Objectives:}\label{objectives-1}

\begin{itemize}
\tightlist
\item
  Introducing components and best practice of lab- and course journals
\item
  Presenting sample wiki-text for Journal entries
\end{itemize}

\subsection{Outcomes:}\label{outcomes-1}

Upon concluding this unit you should be able to \ldots{}

\begin{itemize}
\tightlist
\item
  Begin a structured course journal on the Student Wiki using proper
  wiki text;
\item
  Write your own journal entries, including media images and code as
  required;
\item
  Cross-reference journal entries with links;
\item
  Link to external sources and deliverables on internal pages as
  appropriate;
\item
  Estimate the time you need for tasks, and develop a habit of improving
  your time-management skills.
\end{itemize}

\subsection{Deliverables:}\label{deliverables-1}

\begin{enumerate}
\def\labelenumi{\arabic{enumi}.}
\tightlist
\item
  Your Journal: Your entire journal will be evaluated at the end of the
  course. Refer to the marking rubrics for details.
\item
  Insights: If you find something particularly noteworthy about this
  unit, make a note in your insights! page.
\end{enumerate}

\BeginKnitrBlock{rmd-caution}
\begin{enumerate}
\def\labelenumi{\arabic{enumi}.}
\tightlist
\item
  Your course journal is a deliverable of this course and it will be
  graded. Therefore all rules regarding plagiarism and other academic
  misconduct apply in full. In particular:
\end{enumerate}

\begin{itemize}
\tightlist
\item
  \textbf{do not include any material from elsewhere without referencing
  it:} We are operating a ``full disclosure'' policy in this course.
  Anything that you did not write yourself, on the spot, must be
  referenced. In particular you need to reference if you are copying
  your own material from other courses.;
\item
  \textbf{do not fabricate material that you are posting in your
  journal.} Fabrication could include things like: modifying results
  produced by your code, describing work that you have not actually
  done, or claiming a time for the journal entry that is not the
  time/date on which it was actually written. All of these are academic
  offences.;
\item
  \textbf{Note:} Only journal entries that were written concurrently
  with the activity they describe will be evaluated for credit.
\item
  \textbf{Note:} All journal pages on the Student Wiki---like all other
  submitted material---must contain a \{\{CC-BY\}\} template.

  \EndKnitrBlock{rmd-caution}
\end{itemize}

\subsection{Prerequisites:}\label{prerequisites}

You need the following preparation before beginning this unit. If you
are not familiar with this material from courses you took previously,
you need to prepare yourself from other information sources:

\begin{itemize}
\tightlist
\item
  `''Inquiry''': The scientific method; evidence based reasoning; how to
  design, execute and document an experiment; Conjecture, hypothesis and
  theory.
\item
  `''Writing''': Basic essay and report writing skills. How to format
  your submitted materials, how to quote, cite and avoid plagiarism.
\item
  This unit builds on material covered in the following prerequisite
  units:

  \begin{itemize}
  \tightlist
  \item
    \protect\hyperlink{wiki}{Wiki Unit}
  \end{itemize}
\end{itemize}

\begin{center}\rule{0.5\linewidth}{\linethickness}\end{center}

Work through this unit, then make your work with the ``Plagiarism'' Unit
the first entry of your Journal!

Computational research embraces the same best-practice principles as any
wet-lab experiment. We ensure our work is reproducible, we take great
care that our conclusions are supported by data, and we keep notes to
document our objectives, activities and how we arrived at our results.
Those notes are more than just a handy collection of information: they
need to become a robust, testable record of activities.

Paper notes are not very useful for bioinformatics work because they
can't be cross-referenced easily with computer files. Ideally,
bioinformatics journals will document results, and link to data files,
code repositories, Webpages and other resources. Thus a technical
solution needs to support incorporating or linking to results, data,
code, workflow scripts, documentation, and much more. In this course, we
use the open source Media Wiki software to support journal keeping.

Here are some alternative applications -- but (!) disclaimer, I myself
don't use any of these (yet):

\begin{enumerate}
\def\labelenumi{\arabic{enumi}.}
\tightlist
\item
  \href{https://evernote.com/}{Evernote} - a web hosted, automatically
  syncing e-notebook.
\item
  \href{http://nixnote.org/}{Nevernote} - the Open Source alternative to
  Evernote.
\item
  \href{https://keep.google.com/}{Google Keep} - if you have a Gmail
  account, you can simply log in here. Grid-based. Seems a bit awkward
  for longer notes. But of course you can also use
  \href{https://drive.google.com/drive/my-drive}{Google Docs}.
\item
  \href{https://www.onenote.com/signin?wdorigin=ondc}{Microsoft OneNote}
  - this sounds interesting and if any one is using this, I'd like to
  hear from you. Syncing across platforms, being able to format contents
  and organize it sounds great.
\item
  \href{https://support.rstudio.com/hc/en-us/articles/200526207-Using-Projects}{RStudio
  projects} - for development-focussed work -- especially (but not
  exclusively) -- in R, an RStudio project may be the right solution to
  keep your code, results, notes, manuscript drafts, literature and
  other assets all in one place. The great benefit is that it can all be
  under version control and it's super easy to share everything with
  colleagues on a team through GitHub Technically, GitHub documents are
  all publicly accessible if they are stored in repositories of free
  accounts - but you can commit binary files, so you can simply keep
  sensitive material in password-protected .zip files or otherwise
  encrypted.. The only downside that I can think of is that it's not
  possible to cross-reference and link to material.
\end{enumerate}

Keeping a record of your activities is a habit, and habits need to be
formed through practice. Is this going to be useful to you? I don't
know, but neither do you unless this habit has been given a credible
chance to form. Therefore we practice keeping journals in this course.
As a welcome side effect, this creates a record of activities for future
reference, and provide a basis for evaluation of your progress at the
end of the course. Keeping a journal will help you work with other
learning units or project components effectively, because this is all
integrated over the entire course, and later units often make use of
earlier results which you should have easily accessible.

\textbf{Remember}: you are writing a lab notebook---not a formal lab
report: a point-form record of your actual activities.\footnote{I have
  come across ``journal entries'' that consist only of copy/pasted
  learning unit objectives\ldots{}} Write such documentation as notes to
your (future) self. Record everything that's necessary, but be light and
agile about your writing.

Write your notes immediately, in parallel with your actual activities,
don't draft them elsewhere and expect to enter and revise them later.
Practice shows that delayed processing of journal notes creates an
unmanageable burden. Therefore notes that are not written concurrently
with the activity will not be considered for credit in this course. This
too is about habit forming. But writing concurrently is so easy: since
all of your computational work is done with a computer, begin every
work-session by opening an editing window for its journal entry. Have
the window open, and immediately record everything of importance. The
Wiki is online, so you can even edit your journal from a library
computer, and even (although it's awkward) from your phone.

Obviously, the first step is to create a journal page in the User space
of the Student Wiki - you have already done this in the Wiki editing
unit.

\section{Header}\label{header}

Write a header and give it a unique number.

This is useful so you can refer to the header number in later text.
Obviously, you should ``hard-code'' the number and not use the Wiki's
automatic section numbering scheme, since the numbers should be stable
over time, not change when you add or delete a section\footnote{If the
  Wiki automatically displays section numbers in its Table of Contents,
  you can turn that off in the preferences.}. It is useful to add any
new contents at the top of the page. Keeping the page in reverse
chronological order, prevents you having to scroll to the bottom of the
page every time you add new material. Note though, that the sections do
not actually have to be in strict chronological order, like we would
have them in a paper notebook. Typically you would number in a decimal
system - like 1, 1.1, 1.2, 2, 3 etc. - so you can easily accomodate
additions.

It may be advantageous to give different subprojects their own numbering
space - by adding a prefix to the section number. This depends on how
related the projects are. Everything you keep on the same page is easy
to find with your browser's search function. But if search results come
from different projects, that may be inconvenient. To decide what to put
on the same page and what should go in different subpages, imagine what
material you would search for and what search terms you might
use\footnote{Media Wiki also has its own search functions that search
  for material everywhere on the Wiki, but this is likely not useful on
  the Student Wiki where many users may be writing about similar things.}.

Incidentally: the material in such a notebook is ``permanent'', since
earlier versions of pages are always available via the history function.
The Wiki never forgets. As well, they are automatically time-stamped.
And that's actually a step beyond paper labnotes.

\section{Objective}\label{objective}

\begin{itemize}
\tightlist
\item
  State the objective.
\item
  In one brief sentence, restate what your activity is supposed to
  achieve.
\end{itemize}

\section{Estimate duration}\label{estimate-duration}

The learning units in this course require you to estimate beforehand how
long you will take, and to record how much time you actually took.
Record your initial estimate (work-hours), how many hours you took, and
how much time elapsed between start and end of your task. Make this a
habit in your future coursework as well as in your future labwork. You
will quickly note that you will become much better at time-management.
The sample journal template that is included below contains wikitext to
format a time estimate.

\section{What to document}\label{what-to-document}

\begin{itemize}
\tightlist
\item
  Document the procedure - Note what you have done, as concisely as
  possible but with sufficient detail. ``What is sufficient detail?''
  The answer is easy: detailed enough so that someone can reproduce what
  you have done. In practice that ``someone'' will often be you,
  yourself, in the future. I hope that you won't be constantly cursing
  your past-self because of omissions!
\item
  Document your results.

  \begin{itemize}
  \tightlist
  \item
    You can distinguish different types of results -
  \end{itemize}

  \begin{enumerate}
  \def\labelenumi{\arabic{enumi}.}
  \tightlist
  \item
    Static data does not change over time and it may be sufficient to
    note a reference to the result. For example, there is no need to
    copy a GenBank record into your documentation, it is sufficient to
    note the accession number, the refSeq ID, or the UniProt ID, or even
    better, to link to the relevant page on the external database
    server.
  \item
    Variable data can change over time. For example the results of a
    BLAST search depend on the sequences in the database. A list of
    similar structures may change as new structures get solved and
    deposited in the PDB database. In principle you want to record such
    data, to be able to reproduce at a later time what your conclusions
    were based on. But be selective in what you record. For example you
    should not paste the entire set of results of a BLAST search into
    your document, but only those matches that were important for your
    conclusions. `''Indiscriminate pasting of irrelevant information
    will make your notes unusable.''' Incidentally, the technology to
    expand and collapse paragraphs that we demonstrated in the Wiki
    editing unit can be put to excellent use to record data but keep it
    out of sight when not needed.
  \item
    Analysis results - The results of sequence analyses, alignments etc.
    in general get recorded in your documentation. Again: be selective.
    Record what is important.
  \end{enumerate}
\end{itemize}

\section{Conclusions}\label{conclusions}

Note your conclusions. - An analysis is not complete unless you conclude
something from the results.

\begin{itemize}
\tightlist
\item
  Are two sequences likely homologues, or not? Just pasting the BLAST
  output is not enough. It's your call - `''record it'''.
\item
  Does your protein contain a signal-sequence or does it not? SignalP
  will give you a probability, but you must make the final call.
\item
  Is a binding site conserved, or not? The programs can only point out
  sections of similarity or dissimilarity. You are the one who
  interprets these numbers in their biological context.
\end{itemize}

The analysis provides the data. In your conclusion you provide the
interpretation of what the data means in the context of your objective.
Were you expecting a signal-sequence but there isn't one? What could
that mean? Sometimes your task will explicitly include to elaborate on
an analysis and conclusion. But this does not mean that when analysis is
not explicitly mentioned, you can skip the interpretation. In general
you can never expect full marks if analysis and conclusions are missing.

\section{Outlook for the next tasks}\label{outlook-for-the-next-tasks}

What's the next step? Note it here. Also include a link to the logically
next entry - this way you can quickly hop through consecutive entries
for a theme.

\section{Cross references}\label{cross-references}

Add cross-references.

Cross-references to other information are supremely valuable as your
documentation grows. It's easy to see how to format a link to a section
of your Wiki-page: just look at the link under the Table of Contents at
the top. But you can also place ``anchors'' for linking anywhere on an
HTML page: just use the following syntax. \textless{}span
id=``\{some-label\}''\textgreater{}\textless{}\textbackslash{}span\textgreater{}
for the anchor, and append \#\{some-label\} to the page URL.

\section{Media}\label{media}

\subsection{Images}\label{images}

\begin{itemize}
\tightlist
\item
  Use discretion when uploading images
\item
  Don't upload irrelevant images, don't upload copyrighted images, keep
  the size reasonable. Prepare your images well
\item
  Don't upload uncompressed screen dumps. Save images in a compressed
  file format on your own computer. Then use the Image Upload link in
  the left-hand menu to upload images. The Wiki will only accept .jpeg,
  .png, .gif, or .svg images.
\item
  Use the correct image types.
\item
  In principle, images can be stored uncompressed as .tiff or .bmp, or
  compressed as .gif or .jpg or .png. .gif is useful for images with
  large, monochrome areas and sharp, high-contrast edges because the LZW
  compression algorithm it uses works especially well on such data; .jpg
  (or .jpeg) is preferred for images with shades and halftones such as
  the structure views you should prepare for several assignments, JPEG
  has excellent application support and is the most versatile general
  purpose image file format currently in use; .tiff (or .tif) is
  preferred to archive master copies of images in a lossless fashion,
  use LZW compression for TIFF files if your system/application supports
  it; The .png format is an open source alternative for lossless,
  compressed images.
\item
  .bmp is not preferred for really anything, it is bloated in its
  (default) uncompressed form and primarily used only because it is
  simple to code and ubiquitous on Windows computers. Accordingly we
  don't support it here.
\end{itemize}

\subsection{Image dimensions and
resolution}\label{image-dimensions-and-resolution}

Stereo images should have equivalent points displayed approximately 6cm
apart. It depends on your monitor how many pixels this corresponds to.
The dimensions of an image are stated in pixels (width x height). My
notebook screen has a native display resolution of 1440 x 900
pixels/23.5 x 21 cm. Therefore a 6cm separation on my notebook
corresponds to approximately 260 pixels. However on my desktop monitor,
260 pixels is 6.7 cm across. And on a high-resolution iPad display, at
227 ppi (pixels per inch), 260 pixels are just 2.9 cm across. If your
assigment or learning unit ask you to prepare stero images: adjust your
images so they are approximately at the right separation and are
approximately 500 to 600 pixels wide. Also, scale your molecules so they
fill the available window and - if you have depth cueing enabled - move
them close to the front clipping plane so the molecule is not just a dim
blob, lost in murky shadows.

Considerations for print (manuscripts etc.) are slightly different: for
print output you can specify the output resolution in dpi (dots per
inch). A typical print resolution is about 300 dpi: 6 cm separation at
300dpi is about 700 pixels. Print images should therefore be about three
times as large in width and height as screen images.

\subsection{Preparation of stereo
views}\label{preparation-of-stereo-views}

\begin{itemize}
\tightlist
\item
  When assignments or leartning units ask you to create images of
  molecules, always create stereo views.
\item
  Keep your images uncluttered and expressive
\item
  Scale the molecular model to fill the available space of your image
  well.
\end{itemize}

Orient views so they illustrate a point you are trying to make.
Emphasize residues that you are writing about with a contrasting
colouring scheme. Add labels, where residue identities are not otherwise
obvious. Turn off side-chains for residues that are not important. The
more you practice these small details, the more efficient you will
become in the use of your tools.

\subsection{Code}\label{code}

Always markup code using the GeSHi extension. This provides syntax
highlighting, which is very useful to read the code. You simply place
the code-block into opening- and closing ``source'' tags, and tell GeSHi
which language it should assume. For R-code this looks like:

\begin{Shaded}
\begin{Highlighting}[]
\ControlFlowTok{for}\NormalTok{ i }\ControlFlowTok{in} \DecValTok{1}\OperatorTok{:}\DecValTok{10}\NormalTok{\{}
\NormalTok{  a =}\StringTok{ }\DecValTok{1}
\NormalTok{\}}
\end{Highlighting}
\end{Shaded}

You can also use GeSHi to markup plain text - (although you can achieve
a similar effect by simply beginning each line with a blank " ``).

\begin{verbatim}
Lorem ipsum dolor sit amet ...
\end{verbatim}

\section{Wikitext Template}\label{wikitext-template}

The section below contains Wiki-markup code that you can copy and paste
for your course journal.

\begin{Shaded}
\begin{Highlighting}[]
\OperatorTok{<!--}\StringTok{ }\NormalTok{HTML comments will not be rendered by the wiki. }\OperatorTok{-}\NormalTok{->}

\ErrorTok{<}\NormalTok{div class=}\StringTok{"b1"}\OperatorTok{>}
\NormalTok{Sample Journal}
\OperatorTok{<}\ErrorTok{/}\NormalTok{div}\OperatorTok{>}

\ErrorTok{<}\OperatorTok{!--}\StringTok{ }\NormalTok{To position the table of contents at a particular position, include }\OperatorTok{-}\NormalTok{->}
\ErrorTok{<}\OperatorTok{!--}\StringTok{ }\NormalTok{the __TOC__ }\StringTok{"magic text"} \OperatorTok{-}\NormalTok{->}

\NormalTok{\{\{Vspace\}\}}

\NormalTok{__TOC__}

\NormalTok{\{\{Vspace\}\}}

\OperatorTok{<!--}\StringTok{ }\NormalTok{BEGIN template }\ControlFlowTok{for}\NormalTok{ one entry }\OperatorTok{-}\NormalTok{->}
\ErrorTok{==}\StringTok{ }\ErrorTok{<}\NormalTok{my activity}\OperatorTok{>}\StringTok{ }\ErrorTok{==}

\NormalTok{;Objective}
\OperatorTok{:}\StringTok{ }\NormalTok{text ...}

\OperatorTok{<}\NormalTok{div class=}\StringTok{"time-estimate"}\OperatorTok{>}
\NormalTok{Time estimated}\OperatorTok{:}\StringTok{ }\NormalTok{XXX h; taken XXX h; date started}\OperatorTok{:}\StringTok{ }\DecValTok{2017}\OperatorTok{-}\NormalTok{MM}\OperatorTok{-}\NormalTok{DD; date completed}\OperatorTok{:}\StringTok{ }\DecValTok{2017}\OperatorTok{-}\NormalTok{MM}\OperatorTok{-}\NormalTok{DD}
\OperatorTok{<}\ErrorTok{/}\NormalTok{div}\OperatorTok{>}

\ErrorTok{===}\NormalTok{Progress}\OperatorTok{==}\ErrorTok{=}

\NormalTok{;Activity }\DecValTok{1}
\OperatorTok{:}\StringTok{ }\NormalTok{Notes ...}

\NormalTok{;Activity }\DecValTok{2}
\OperatorTok{:}\StringTok{ }\NormalTok{Notes ...}

\OperatorTok{==}\ErrorTok{=}\StringTok{ }\NormalTok{Conclusion and outlook}\OperatorTok{==}\ErrorTok{=}

\NormalTok{Describe ...}

\NormalTok{\{\{Vspace\}\}}

\OperatorTok{<!--}\StringTok{ }\NormalTok{END of template }\ControlFlowTok{for}\NormalTok{ one entry }\OperatorTok{-}\NormalTok{->}


\ErrorTok{<}\OperatorTok{!--}\StringTok{ }\NormalTok{BEGIN sample entry }\OperatorTok{-}\NormalTok{->}

\ErrorTok{==}\StringTok{ }\NormalTok{(Example Journal Entry) Explore Cytoscape }\OperatorTok{==}

\NormalTok{;Objective}
\OperatorTok{:}\StringTok{ }\NormalTok{Write R code to generate a random network. Explore it with Cytoscape.}

\OperatorTok{<}\NormalTok{div class=}\StringTok{"time-estimate"}\OperatorTok{>}
\NormalTok{Time estimated}\OperatorTok{:}\StringTok{ }\DecValTok{2}\NormalTok{ h; taken }\DecValTok{4}\NormalTok{ h; date started}\OperatorTok{:}\StringTok{ }\DecValTok{2017}\OperatorTok{-}\DecValTok{09}\OperatorTok{-}\DecValTok{16}\NormalTok{; date completed}\OperatorTok{:}\StringTok{ }\DecValTok{2017}\OperatorTok{-}\DecValTok{09}\OperatorTok{-}\DecValTok{18}
\OperatorTok{<}\ErrorTok{/}\NormalTok{div}\OperatorTok{>}

\ErrorTok{===}\NormalTok{Progress}\OperatorTok{==}\ErrorTok{=}

\NormalTok{;R code }\ControlFlowTok{for}\NormalTok{ a random network.}
\OperatorTok{:}\StringTok{ }\NormalTok{Cytoscape can read networks }\ControlFlowTok{in}\NormalTok{ [http}\OperatorTok{:}\ErrorTok{//}\NormalTok{wiki.cytoscape.org}\OperatorTok{/}\NormalTok{Cytoscape_User_Manual}\OperatorTok{/}\NormalTok{Network_Formats}\CommentTok{#SIF_format SIF format]. Wrote the following code to generate N random nodes with random interactions and write them to file.}

\OperatorTok{<}\NormalTok{div class=}\StringTok{"toccolours mw-collapsible mw-collapsed"}\NormalTok{ style=}\StringTok{"width:800px"}\OperatorTok{>}
\NormalTok{My R code below ...}
\OperatorTok{<}\NormalTok{div class=}\StringTok{"mw-collapsible-content"}\OperatorTok{>}

\ErrorTok{<}\NormalTok{source lang=}\StringTok{"R"}\OperatorTok{>}
\NormalTok{N <-}\StringTok{ }\DecValTok{15}  \CommentTok{# number of nodes}
\NormalTok{nMin <-}\StringTok{ }\DecValTok{1} \CommentTok{# minimum number of edges}
\NormalTok{nMax <-}\StringTok{ }\DecValTok{5} \CommentTok{# maximum number of edges}
\NormalTok{OUT <-}\StringTok{ "myRandomSIF.txt"}


\CommentTok{# Create N random strings from three uppercase characters}
\CommentTok{# First make more than we need, because some might not be unique ...}
\NormalTok{nodes <-}\StringTok{ }\KeywordTok{character}\NormalTok{(}\DecValTok{2}\OperatorTok{*}\NormalTok{N)}

\KeywordTok{set.seed}\NormalTok{(}\DecValTok{161803}\NormalTok{)}
\ControlFlowTok{for}\NormalTok{ (i }\ControlFlowTok{in} \DecValTok{1}\OperatorTok{:}\NormalTok{(}\DecValTok{2}\OperatorTok{*}\NormalTok{N)) \{}
\NormalTok{  nodes[i] <-}\StringTok{ }\KeywordTok{paste0}\NormalTok{(}\KeywordTok{sample}\NormalTok{(LETTERS, }\DecValTok{1}\NormalTok{),}
                     \KeywordTok{sample}\NormalTok{(LETTERS, }\DecValTok{1}\NormalTok{),}
                     \KeywordTok{sample}\NormalTok{(LETTERS, }\DecValTok{1}\NormalTok{))}
\NormalTok{\}}
\KeywordTok{set.seed}\NormalTok{(}\OtherTok{NULL}\NormalTok{)}

\NormalTok{nodes <-}\StringTok{ }\KeywordTok{unique}\NormalTok{(nodes) }\CommentTok{# throw away duplicates ...}
\NormalTok{nodes <-}\StringTok{ }\NormalTok{nodes[}\DecValTok{1}\OperatorTok{:}\NormalTok{N]    }\CommentTok{# ... and keep only N}

\CommentTok{# Create SIF output}
\NormalTok{mySIF <-}\StringTok{ }\KeywordTok{character}\NormalTok{()}

\KeywordTok{set.seed}\NormalTok{(}\DecValTok{112358}\NormalTok{)}
\ControlFlowTok{for}\NormalTok{ (myA }\ControlFlowTok{in}\NormalTok{ nodes) \{  }\CommentTok{# for each node}
  \ControlFlowTok{for}\NormalTok{ (i }\ControlFlowTok{in} \DecValTok{1}\OperatorTok{:}\KeywordTok{sample}\NormalTok{(nMin}\OperatorTok{:}\NormalTok{nMax, }\DecValTok{1}\NormalTok{)) \{ }\CommentTok{# for a random number of interactors ...}
\NormalTok{    myB <-}\StringTok{ }\KeywordTok{sample}\NormalTok{(nodes, }\DecValTok{1}\NormalTok{)           }\CommentTok{# ... randomly choose interactor ...}
                                      \CommentTok{# ... create record and apend to mySIF}
\NormalTok{    mySIF <-}\StringTok{ }\KeywordTok{c}\NormalTok{(mySIF, }\KeywordTok{sprintf}\NormalTok{(}\StringTok{"%s}\CharTok{\textbackslash{}t}\StringTok{pp}\CharTok{\textbackslash{}t}\StringTok{%s"}\NormalTok{, myA, myB)) }\CommentTok{# myA, "pp", myB}
                                                       \CommentTok{# separated by tabs}
\NormalTok{  \}}
\NormalTok{\}}
\KeywordTok{set.seed}\NormalTok{(}\OtherTok{NULL}\NormalTok{)}

\CommentTok{# write mySIF to file}
\KeywordTok{writeLines}\NormalTok{(mySIF, OUT)}
\OperatorTok{<}\ErrorTok{/}\NormalTok{source}\OperatorTok{>}

\ErrorTok{</}\NormalTok{div}\OperatorTok{>}
\StringTok{  }\ErrorTok{</}\NormalTok{div}\OperatorTok{>}

\NormalTok{;Install Cytoscape}
\OperatorTok{:}\StringTok{ }\NormalTok{Straightforward installation of }\OperatorTok{<}\NormalTok{b}\OperatorTok{>}\NormalTok{Cytoscape }\FloatTok{3.5}\NormalTok{.}\DecValTok{1}\OperatorTok{<}\ErrorTok{/}\NormalTok{b}\OperatorTok{>}\StringTok{ }\NormalTok{from http}\OperatorTok{:}\ErrorTok{//}\NormalTok{www.cytoscape.org}

\NormalTok{;Visualize and explore options}
\OperatorTok{*}\ErrorTok{:}\NormalTok{Worked through an introductory tutorial from http}\OperatorTok{:}\ErrorTok{//}\NormalTok{opentutorials.cgl.ucsf.edu}\OperatorTok{/}\NormalTok{index.php}\OperatorTok{/}\NormalTok{Tutorial}\OperatorTok{:}\NormalTok{Introduction_to_Cytoscape_}\FloatTok{3.1}\OperatorTok{-}\NormalTok{part2 Noteworthy}\OperatorTok{:}\StringTok{ }\NormalTok{nodes can be coloured by attribute values.}
\OperatorTok{*}\ErrorTok{:}\NormalTok{Loaded my own SIF file with the File → Import → Network → File option. Explored various layout options and visualization styles. Manually added nodes and edges with the options one gets when ctrl}\OperatorTok{-}\NormalTok{clicking on the canvas.}

\NormalTok{\{\{Vspace\}\}}

\OperatorTok{==}\ErrorTok{=}\StringTok{ }\NormalTok{Conclusion and outlook}\OperatorTok{==}\ErrorTok{=}

\ErrorTok{*}\StringTok{ }\NormalTok{Creating layouts, applying styles, adding nodes and edges with Cytoscape seems straightforward and creates evocative images.}
\OperatorTok{*}\StringTok{ }\NormalTok{Analyzing and interpreting the network needs more thought. Need to explore ideas and tools }\ControlFlowTok{for}\NormalTok{ network analysis }\ControlFlowTok{next} \OperatorTok{-}\StringTok{ }\NormalTok{e.g. clustering tools, enrichment tools, motif discovery ...}
\OperatorTok{*}\StringTok{ }\NormalTok{Next}\OperatorTok{:}\StringTok{ }\NormalTok{write some code to create attribute values }\ControlFlowTok{for}\NormalTok{ nodes, import the values, and have Cytoscape style the nodes accordingly.}

\NormalTok{\{\{Vspace\}\}}

\OperatorTok{----}

\ErrorTok{<}\OperatorTok{!--}\StringTok{ }\NormalTok{Footnotes }\OperatorTok{-}\NormalTok{->}
\ErrorTok{==}\NormalTok{Notes and References}\OperatorTok{==}
\ErrorTok{<}\NormalTok{references }\OperatorTok{/}\ErrorTok{>}

\NormalTok{\{\{Vspace\}\}}

\OperatorTok{<!--}\StringTok{ }\NormalTok{Category and Footer }\OperatorTok{-}\NormalTok{->}
\NormalTok{[[Category}\OperatorTok{:}\NormalTok{BCH441}\OperatorTok{-}\NormalTok{2019_Journal]]}

\OperatorTok{----}

\NormalTok{\{\{CC}\OperatorTok{-}\NormalTok{BY\}\}}

\OperatorTok{<!--}\StringTok{ }\NormalTok{END }\OperatorTok{-}\NormalTok{->}
\end{Highlighting}
\end{Shaded}

\section{Self-evaluation}\label{self-evaluation-1}

\section{Further reading, links and
resources}\label{further-reading-links-and-resources-1}

\textbf{If in doubt, ask!} If anything about this learning unit is not
clear to you, do not proceed blindly but ask for clarification. Post
your question on the course mailing list: others are likely to have
similar problems. Or send an email to your instructor.

\BeginKnitrBlock{rmd-original-history}
\textbf{Author}: Boris Steipe
\href{mailto:boris.steipe@utoronto.ca}{\nolinkurl{boris.steipe@utoronto.ca}}
\textbf{Created}: 2017-08-05 \textbf{Modified}: 2019-01-05 Version: 1.3
\textbf{Version history}: 1.3 Emphasize habit forming and cuncurrent
editing. Note on license. 1.2 Make time tags mandatory; warn against
fabrication. 1.1 Add GeSHi example 1.0 First live version 0.1 First stub
\EndKnitrBlock{rmd-original-history}

\subsection{Updated Revision history}\label{updated-revision-history-1}

\begin{tabular}{l|l|l|l}
\hline
Revision & Author & Date & Message\\
\hline
fa320fb & Ruth Isserlin & 2019-12-23 & modified edit settings on bookdown configurationmodified edit settings on bookdown configuration\\
\hline
f56a24c & Ruth Isserlin & 2019-12-23 & Added new git info to each fileAdded new git info to each file (in addition to the original version history copied over from Boris's wiki).\\
\hline
8950904 & Ruth Isserlin & 2019-12-22 & Initial check in of converted wiki pages from Boris Steipe's bcb420 course material pagewiki pages were converted to bookdown and formatted to the bookdown format\\
\hline
\end{tabular}

\chapter{Insights!}\label{insights}

(Instructions for authoring a collection of course-related insights)

\section{Overview}\label{overview-2}

\subsection{Abstract:}\label{abstract-2}

In paralell with your other work, you will maintain an insights! page on
which you collect valuable insights and learning experiences of the
course. Through this you ask yourself: what does this material mean -
for the field, and for myself.

\subsection{Objectives:}\label{objectives-2}

\begin{itemize}
\tightlist
\item
  Self-reflect on the meaning of the course contents.
\item
  Question and develop your own attitudes about the course and the
  field.
\item
  Concisely record essential insights.
\end{itemize}

\subsection{Outcomes:}\label{outcomes-2}

\begin{itemize}
\tightlist
\item
  Through your continuos work on the insights! page you have \ldots{}
\item
  Developed an understanding of what makes a learning activity
  meaningful;
\item
  Identified and summarized the most meaningful learning experiencers in
  the course;
\item
  Authored a collection of insights that can remind you of the course
  experience.
\end{itemize}

\subsection{Deliverables:}\label{deliverables-2}

\begin{enumerate}
\def\labelenumi{\arabic{enumi}.}
\tightlist
\item
  Your insights! page that is linked from your user page on the Student
  Wiki.
\end{enumerate}

\subsection{Prerequisites:}\label{prerequisites-1}

This unit builds on material covered in the following prerequisite
units:

\begin{itemize}
\tightlist
\item
  \protect\hyperlink{journal}{Journal Unit}
\end{itemize}

\subsection{Evaluation}\label{evaluation}

Your insights! page will be evaluated by the instructor at the end of
the course for a maximum of 5 marks. Find the evaluation rubric
\href{http://steipe.biochemistry.utoronto.ca/abc/index.php/ABC-Rubrics\#Insights}{here}.

\section{Insights}\label{insights}

It's rare that a fundamental insight is formulated so memorably, but it
is so very valuable to step back and ask yourself: what was really
important about what I just did? What is the one thing I should
remember? Maintaining such a high-level perspective goes a long way to
ensure that your learning effort turns into something valuable. Begin an
insights! page that collects your most important ideas about this term.
Be concise and succinct - focus on the essence of your experience. And
if you can put it into rhyme - all the more memorable.

\begin{itemize}
\tightlist
\item
  Consider the marking
  \href{http://steipe.biochemistry.utoronto.ca/abc/index.php/ABC-Rubrics\#insights.21}{rubric}:
  continuous engagement is important, as well as successfully grasping
  the essence of what you are doing. You are welcome to study other's
  pages, and to quote, but you must give credit where appropriate. Be
  very careful not to plagiarize.
\end{itemize}

\section{Task}\label{task-1}

\BeginKnitrBlock{rmd-task}
\begin{itemize}
\tightlist
\item
  Make a link to a new insights! page as a subpage of your user page on
  the Student Wiki.\\
\item
  Make sure the page is created in the correct namespace, not in the
  main space of the Wiki.

  \begin{itemize}
  \tightlist
  \item
    Edit the page to give it a meaningful structure.
  \item
    Add a category tag {[}{[}Category:Insights!{]}{]} to the page.
  \item
    As with all course-material, add a \{\{CC-BY\}\} tag to the bottom
    of the page to include a Creative Commons license.
  \item
    Whenever you make an interesting observation about the course, the
    material, or something else that you find inspiring, add it to your
    page.
  \item
    You can write only your insight, but it may be more valuable to add
    a bit of context. You may certainly review and revise, strenghten or
    discard insights throughout the term. And you are encouraged to
    study others opinions. However: be mindful that your insights are
    your own, personal, unique reflections. I could imagine one or two
    such insights per week.

    \EndKnitrBlock{rmd-task}
  \end{itemize}
\end{itemize}

\section{Template}\label{template}

You can use the following template of Wikitext for your insights or
design your own. Take care: formatting counts for evaluation.:

\begin{Shaded}
\begin{Highlighting}[]
\OperatorTok{==}\ErrorTok{==<}\NormalTok{Insight Title}\OperatorTok{>=}\ErrorTok{===}
\StringTok{ }
\ErrorTok{<}\NormalTok{context... }\OperatorTok{>}
\NormalTok{;}\OperatorTok{<}\NormalTok{insight}\OperatorTok{>}
\ErrorTok{:<}\NormalTok{date}\OperatorTok{>}
\StringTok{ }
\NormalTok{\{\{Vspace\}\}}
\end{Highlighting}
\end{Shaded}

\section{Further reading, links and
resources}\label{further-reading-links-and-resources-2}

\textbf{If in doubt, ask!} If anything about this learning unit is not
clear to you, do not proceed blindly but ask for clarification. Post
your question on the course mailing list: others are likely to have
similar problems. Or send an email to your instructor.

\BeginKnitrBlock{rmd-original-history}
\textbf{Author}: Boris Steipe
\href{mailto:boris.steipe@utoronto.ca}{\nolinkurl{boris.steipe@utoronto.ca}}
\textbf{Created}: 2017-08-05 \textbf{Modified}: 2017-08-07 Version: 1.0
\textbf{Version history}: 1.0 Live version 0.1 First stub
\EndKnitrBlock{rmd-original-history}

\subsection{Updated Revision history}\label{updated-revision-history-2}

\begin{tabular}{l|l|l|l}
\hline
Revision & Author & Date & Message\\
\hline
f56a24c & Ruth Isserlin & 2019-12-23 & Added new git info to each fileAdded new git info to each file (in addition to the original version history copied over from Boris's wiki).\\
\hline
8950904 & Ruth Isserlin & 2019-12-22 & Initial check in of converted wiki pages from Boris Steipe's bcb420 course material pagewiki pages were converted to bookdown and formatted to the bookdown format\\
\hline
\end{tabular}

\chapter{Plagiarism and academic integrity}\label{plagiarism}

(Plagiarism, proper citing, steering clear of academic misconduct, rules
for collaboration)

\section{Overview}\label{overview-3}

\subsection{Abstract:}\label{abstract-3}

In this course we operate a Full Disclosure Policy for attribution. This
means everything that is not your own, original idea must be identified
and properly attributed.\footnote{This includes journal articles and
  books, obviously, but also blogs, discussions on StackOverflow, Github
  gists, and especially course material of this and other courses, your
  peer's notes and journal entries. and material you have produced
  previously.}

\subsection{Objectives:}\label{objectives-3}

This unit will:

\begin{itemize}
\tightlist
\item
  point out the rationale for proper attribution;
\item
  define the ``Full Disclosure Policy'' for attribution in my courses;
\item
  introduce the resources available to you to avoid plagiarism and other
  academic misconduct.
\end{itemize}

\subsection{Outcomes:}\label{outcomes-3}

After working through this unit you:

\begin{itemize}
\tightlist
\item
  are aware how academic offences can be avoided with a full-disclosure
  policy;
\item
  are familiar with proper citation formats for journal and Web
  contents;
\item
  ensure that all your activities in this course and elsewhere are in
  accordance with the letter of the published policies, and the spirit
  of scientific integrity.
\end{itemize}

\subsection{Deliverables:}\label{deliverables-3}

\textbf{Time management}: Before you begin, estimate how long it will
take you to complete this unit. Then, record in your course journal: the
number of hours you estimated, the number of hours you worked on the
unit, and the amount of time that passed between start and completion of
this unit.

\textbf{Journal}: Document your progress in your Course Journal. Some
tasks may ask you to include specific items in your journal. Don't
overlook these.

\textbf{Insights}: If you find something particularly noteworthy about
this unit, make a note in your insights! page.

\section{Plagarism}\label{plagarism}

\textbf{Make this unit the first entry of your Journal!}

This is the age of open source, boundless mashup, awesome reposts and
instant repurposing of information. It might seem that our rules of
referencing and citation are just another academic anachronism. After
all, we all copy from \href{https://stackoverflow.com/}{stack overflow},
right?

No - actually: wrong for two reasons. One: information is not any longer
a commodity that increases in value if its supply is artificially
constrained. Rather the value of information in academia - our common
currency - is now ``mindshare'', and mindshare cannot grow without
attribution. And two: part of any course at UofT is its ``summative
assessment'': we mark submissions to evaluate the aptitude and
achievements of students. That can only be done if original thinking by
the student is clearly identified, and distinguished from merely
repeating other's thoughts.

But let's face it: \textbf{UofT has a plagiarism problem}. This has
gotten worse over the past years - and it seems worse in the CS realms
than in the domains of the life sciences. Yet, even if all your peers
think no one cares about missing attributions, that doesn't make it
somehow right: ethics is not a result of opinion polls. No matter how
socially acceptable plagiarism has become, no matter how many others do
``it'', no matter how many likes or upvotes or retweets a ``No Big
Deal!'' post attracts, unethical behaviour is wrong. It goes against
everything we stand for as scientists. And it is corrosive, not just for
your community, but first and foremost for yourself.

In this course we operate a \textbf{Full Disclosure Policy}. That
doesn't mean you can't get good data and examples from wherever you find
them - on the contrary I absolutely expect you to hunt everywhere for
information and examples: biostars, stack, RBloggers, even reddit
(sometimes). There is great value in finding how others have addressed a
problem, or divide up and organize a particular topic, and compiling the
knowledge of the entire community is a great starting point for
excellent work. But (a) this process has to be transparent, and, indeed
you need to develop and document a sense of pride in mastering this art
and attributing the contributions of your sources, and (b) compiling
information does not substitute for your understanding of the material
that you are presenting.

Regardless whether you are reusing, quoting, paraphrasing, summarizing
or following someone else's structure or advice: reference it. You can
never reference too much, but if you don't reference enough you are
probably committing an academic offence and I am obliged by University
Policy to take the issue to the
\href{https://www.artsci.utoronto.ca/current/academic-advising-and-support/student-academic-integrity}{Office
of Student Academic Integrity (OSAI)}. Regardless whether you are
writing an assignment, updating your journal, uploading code, replying
to posts on the mailing list - for anything that is submitted for
credit, directly or indirectly: make sure all your references are
complete. The principle is quite simple:

\begin{rmd-caution}
** Full disclosure policy for this course:**

\begin{itemize}
\tightlist
\item
  \textbf{If it's not your own, new idea, it has a source.}
\item
  \textbf{All sources must be referenced.}
\end{itemize}
\end{rmd-caution}

\section{How to reference}\label{how-to-reference}

You probably have seen resources that refer to other's observations or
opinions, and teach you to reference in manuscripts and essays in the
life sciences and the humanities. These are generally less relevant for
the kind of work that we do, and perhaps this is one of the reasons for
poor uptake. Indeed, most of the writing in our courses is informal, and
it may not be obvious how to properly embed citations in the flow of the
narrative. The solution is to thouroughly \textbf{contextualize your
attributions} with statements such as:

\begin{itemize}
\tightlist
\item
  ``inspired by\ldots{}'',
\item
  ``based on\ldots{}'',
\item
  ``according to \ldots{}'',
\item
  ``following \ldots{}'',
\item
  ``see also \ldots{}'',
\item
  etc. as appropriate.
\end{itemize}

Some specific points to consider:

\begin{enumerate}
\def\labelenumi{\arabic{enumi}.}
\tightlist
\item
  On Wiki pages, use the text tag to organize your citations.
\item
  In R code, put your citations directly into the code in a comment.
\item
  The URL of an image on the Internet is not by itself an adequate
  reference. Author and context must be shown.
\item
  All links in references must be to the original source, not to e.g.~a
  blog post about the source.
\item
  Code that is copied from e.g.~stack overflow or similar must be
  referenced with a link to the post AND the name of the author.
\item
  Use the APA citation format for this course.
\end{enumerate}

\section{When and what to reference}\label{when-and-what-to-reference}

Some material may be ``common knowledge''. Obviously, you don't need to
cite the resource that taught you how to write a for-loop, for example.
Where to draw the line? If in doubt, ask, discuss, seek consensus in the
community and in class.

\begin{itemize}
\tightlist
\item
  Example code from an R package vignette that you adapt and reuse must
  be referenced.
\item
  Code that you take from a different course must be referenced.
\item
  Code that you take from this course must be referenced.
\item
  Code that you have translated from a different language must be
  referenced.
\item
  Code that you have jointly developed with a classmate must acknowledge
  the contribution.
\item
  Code that you have copied from a classmate's work on the Student Wiki
  must be referenced.
\item
  Code that is open source must be referenced.
\item
  Code that is in the public domain must be referenced. This one is
  particularly troublesome: some authors put their code into the public
  domain, and state that you are free to use it without any copyright
  restrictions and without need for acknowledgement. But this can give
  rise to a misunderstanding: it only refers to the legal status as far
  as reuse is concerned, \textbf{it says nothing about authorship}.
  Obviously: just because someone graciously allowed you to use their
  idea, that does not mean that you are the author.
\end{itemize}

\section{Reusing previous material}\label{reusing-previous-material}

A second mistake that has brought students to the Dean's office more
than once is re-use of material from previous courses. This is a simple
one: \textbf{you can't get academic credit for the same material twice}.
This means: if you have already submitted something for a different
course elsewhere, or for a different assessment in the same course, it
is no longer an original contribution. Of course you can cite your own
work and then reuse it - if it's useful, bravo - good for you. But you
have to be upfront about it, and apply the Full Disclosure Policy in
spirit. Again: if in doubt, ask for advice.

\section{\texorpdfstring{``Adjusting''
results}{Adjusting results}}\label{adjusting-results}

It sometimes happens that a piece of code you are submitting won't run.
It just won't. You can't see the mistake, it's three in the morning and
you just can't take it anymore. It's just a small variation from the
spec - and you can easily fix the output by hand.

So you edit a few lines in a printout, or a few cells in a spreadsheet,
and submit that result. All good, right?

No. Not good at all. You have just falsified your code output. In terms
of academic misconduct this is called ``concoction''. And it's pretty
high on the list of things that will make for a very bad day.

Do not ever change code output by hand. If an assignment asks you to
submit code and results, \textbf{the exact code you submit must generate
the exact output that is claimed for it}. Obvioulsy, there will be
assessment scripts that will verify that. And when the assessment script
signals a discrepancy, that will set off a process \ldots{}

Above, I've highlighted a few issues that I have come accross in past
courses. Below, are extensive resources that will help you work better.
Go and read them.

\section{Task}\label{task-2}

\BeginKnitrBlock{rmd-task}
\begin{itemize}
\item
  Visit the following sites and read the material carefully:

  \begin{enumerate}
  \def\labelenumi{\arabic{enumi}.}
  \tightlist
  \item
    \href{https://www.academicintegrity.utoronto.ca/smart-strategies/}{Smart
    Strategies (OSAI)}
  \item
    \href{http://advice.writing.utoronto.ca/using-sources/how-not-to-plagiarize/}{How
    Not to Plagiarize (UofT Writing Advice)}
  \item
    \href{http://advice.writing.utoronto.ca/using-sources/quotations/}{Quoting
    (UofT Writing Advice)} Mostly relevant for essays, but also for your
    own journal: when to quote, when to paraphrase, when to summarize.
  \item
    \href{http://advice.writing.utoronto.ca/using-sources/paraphrase/}{Paraphrasing
    and Summarizing (UofT Writing Advice)} Illustrating the distinction
    between illegitimate and proper paraphrase, and summary by example.
  \item
    \href{http://advice.writing.utoronto.ca/using-sources/documentation/}{Standard
    Documentation formats (UofT Writing Advice)} This includes a concise
    overview on citing electronic sources.
  \end{enumerate}
\item
  Then - for your own reference - \textbf{put a model of the following
  three types of references into your journal}:

  \begin{itemize}
  \tightlist
  \item
    a procedure in the methods section of a journal article, as you
    would cite it in a technical report;
  \item
    a piece of code you found in a StackOverflow article, as you would
    put it as a comment into computer code;
  \item
    some contents from a classmate's journal that you incoporate into
    your own journal.

    \EndKnitrBlock{rmd-task}
  \end{itemize}
\end{itemize}

\section{A final note}\label{a-final-note}

If you have any questions about these policies, if you would like to
start a discussion, or if you can share some anecdotes: post on the
mailing list. We need to make these policies alive and meaningful,
otherwise we'll see mistake after mistake, tears, anguish and missed
graduations. No more!

\section{Further reading, links and
resources}\label{further-reading-links-and-resources-3}

\href{https://governingcouncil.utoronto.ca/secretariat/policies/code-behaviour-academic-matters-july-1-2019}{Code
of Behaviour on Academic Matters The policy}. (Governing Council of the
University of Toronto.)

\textbf{If in doubt, ask!} If anything about this learning unit is not
clear to you, do not proceed blindly but ask for clarification. Post
your question on the course mailing list: others are likely to have
similar problems. Or send an email to your instructor.

\BeginKnitrBlock{rmd-original-history}
\textbf{Author}: Boris Steipe
\href{mailto:boris.steipe@utoronto.ca}{\nolinkurl{boris.steipe@utoronto.ca}}
\textbf{Created}: 2017-08-05 \textbf{Modified}: 2019-01-15 Version:
1.1.1 \textbf{Version history}: 1.1.1 Clarification: this courses
material falls under the same policy 1.1 Update to ``Full Disclosure
Policy'' 1.0 Completed to first live version 0.2 Links to FAS material
0.1 Material collected from previous tutorial
\EndKnitrBlock{rmd-original-history} \#\#\# Updated Revision history

\begin{tabular}{l|l|l|l}
\hline
Revision & Author & Date & Message\\
\hline
f56a24c & Ruth Isserlin & 2019-12-23 & Added new git info to each fileAdded new git info to each file (in addition to the original version history copied over from Boris's wiki).\\
\hline
8950904 & Ruth Isserlin & 2019-12-22 & Initial check in of converted wiki pages from Boris Steipe's bcb420 course material pagewiki pages were converted to bookdown and formatted to the bookdown format\\
\hline
\end{tabular}

\subsection{Footnotes:}\label{footnotes-1}

\chapter{Data Backup Best Practice}\label{backup}

(Data backup technologies and best practice)

\section{Overview}\label{overview-4}

\subsection{Abstract:}\label{abstract-4}

\subsection{Objectives:}\label{objectives-4}

This unit will provide a brief introduction to backup problems and
methods to solve them.

\subsection{Outcomes:}\label{outcomes-4}

After working through this unit you \ldots{}

\begin{itemize}
\tightlist
\item
  have a sensible and effective strategy for backing up your computer;
\item
  have demonstrated that you can recover data.
\end{itemize}

\subsection{Deliverables:}\label{deliverables-4}

\textbf{Time management}: Before you begin, estimate how long it will
take you to complete this unit. Then, record in your course journal: the
number of hours you estimated, the number of hours you worked on the
unit, and the amount of time that passed between start and completion of
this unit.

\textbf{Journal}: Document your progress in your Course Journal. Some
tasks may ask you to include specific items in your journal. Don't
overlook these.

\textbf{Insights}: If you find something particularly noteworthy about
this unit, make a note in your insights! page.

\section{Data backups}\label{data-backups}

When was the last time you made a full backup of your computer's
hard-drive? Too long ago? I thought so.

\begin{rmd-task}
\begin{enumerate}
\def\labelenumi{\arabic{enumi}.}
\tightlist
\item
  Backup your hard-drive now.
\end{enumerate}
\end{rmd-task}

Risk is probability times damage. The annualized probability of hard
drive failure is on the order of 3\%. This means about three of your
classmates in this course will lose all their data this year. It is a
bit better for solid state drives (SSD), perhaps only 0.3 failures per
year. But what is the damage? All your essays, coursework, all the
pictures on your computer, that email from your first love\ldots{} There
is no question you need to have plans in place to protect your data.
After all, storage failure is not a question of if, but when.

Enterprise-scale data in bioinformatics labs need dedicated solution and
that is actually an increasing problem. For your small scale, personal
backup needs you have a variety of options;

\begin{itemize}
\tightlist
\item
  Cloud backup may be bandwidth limited;
\item
  Mac OS X has things well covered with thier Time machine / Time
  capsule, I don't know what the best equivalent is for other systems;
\item
  off-machine storage in removable disk is a questionable startegy
  because everything that is not fully automatic is liable to fall
  victim to complecancy;
\item
  Whenever possible, make differential backups. And remember: NO backup
  is a backup unless recovery of data has been tested and shown to work.
\end{itemize}

\section{Task:}\label{task-3}

\begin{rmd-task}
\begin{itemize}
\tightlist
\item
  Use this Google Search for links to recent articles about backup
  options.
\item
  Read a few articles that are relevant for your computer.
\item
  Decide on a backup strategy for your computer.
\item
  Implement your strategy.
\item
  Create a test file.
\item
  Backup your computer.
\item
  Delete your test file.
\item
  Recreate the file from your last backup.
\end{itemize}
\end{rmd-task}

\section{Self-evaluation}\label{self-evaluation-2}

\section{Notes}\label{notes}

\section{Further reading, links and
resources}\label{further-reading-links-and-resources-4}

Use this Google Search for links to recent articles about backup
options.

\textbf{If in doubt, ask!} If anything about this learning unit is not
clear to you, do not proceed blindly but ask for clarification. Post
your question on the course mailing list: others are likely to have
similar problems. Or send an email to your instructor.

\BeginKnitrBlock{rmd-original-history}
\textbf{Author}: Boris Steipe
\href{mailto:boris.steipe@utoronto.ca}{\nolinkurl{boris.steipe@utoronto.ca}}
\textbf{Created}: 2017-08-05 \textbf{Modified}: 2017-09-11 Version: 1.0
\textbf{Version history}: 1.0 Completed to first live version 0.2 Begin
development from legacy material; points only 0.1 Material collected
from previous tutorial
\EndKnitrBlock{rmd-original-history}

\subsection{Updated Revision history}\label{updated-revision-history-3}

\begin{tabular}{l|l|l|l}
\hline
Revision & Author & Date & Message\\
\hline
f56a24c & Ruth Isserlin & 2019-12-23 & Added new git info to each fileAdded new git info to each file (in addition to the original version history copied over from Boris's wiki).\\
\hline
8950904 & Ruth Isserlin & 2019-12-22 & Initial check in of converted wiki pages from Boris Steipe's bcb420 course material pagewiki pages were converted to bookdown and formatted to the bookdown format\\
\hline
\end{tabular}

\chapter{Cargo Cult Science}\label{cargocult}

(Cargo Cult science, Cargo Cult bioinformatics)

\section{Overview}\label{overview-5}

\subsection{Abstract:}\label{abstract-5}

Not all activities lead to valuable outcomes and ``Cargo Cult Science''
is an important metaphor for a class of conceptual problems that are the
hallmark of ``poor science''. The particular issue is that activites are
causally disconnected from their claimed beneficial outcomes.

\subsection{Objectives:}\label{objectives-5}

This unit will:

\begin{itemize}
\tightlist
\item
  introduce the metaphor of a ``Cargo Cult'', as applied to
  bioinformatics and science in general;
\item
  illustrate with examples;
\item
  discuss principles to avoid the problem.
\end{itemize}

\subsection{Outcomes:}\label{outcomes-5}

After working through this unit you:

\begin{itemize}
\tightlist
\item
  can identify ``Cargo Cult''-type issues;
\item
  have contributed an example to our collection, and/or;
\item
  have contributed to the discussion in our collection;
\item
  are able to critically evaluate projects and activities regarding
  whether they can contribute to thier claimed value;
\item
  are able to propose improvements.
\end{itemize}

\subsection{Deliverables:}\label{deliverables-5}

\textbf{Time management}: Before you begin, estimate how long it will
take you to complete this unit. Then, record in your course journal: the
number of hours you estimated, the number of hours you worked on the
unit, and the amount of time that passed between start and completion of
this unit.

\textbf{Journal}: Document your progress in your Course Journal. Some
tasks may ask you to include specific items in your journal. Don't
overlook these.

\textbf{Insights}: If you find something particularly noteworthy about
this unit, make a note in your insights! page.

\subsection{Prerequisites:}\label{prerequisites-2}

You need the following preparation before beginning this unit. If you
are not familiar with this material from courses you took previously,
you need to prepare yourself from other information sources:

Inquiry: The scientific method; evidence based reasoning; how to design,
execute and document an experiment; Conjecture, hypothesis and theory.

\section{Cargo Cult}\label{cargo-cult}

The concept of
\href{https://en.wikipedia.org/wiki/Cargo_cult_science}{Cargo cult
science} was popularized by
\href{https://en.wikipedia.org/wiki/Richard_Feynman}{Richard Feynman} in
the 1974
\href{http://calteches.library.caltech.edu/51/2/CargoCult.htm}{Caltech
Commencement address}. In a nutshell, Feynman points out how scientific
practices that lack ``scientific integrity'' are similar to the
activities of a premodern cult in the South Sea islands that developed
rituals for attracting goods-bearing supply airplanes by building mock
airports.

The essence of Cargo Cult is not merely poor science. What makes it
``Cargo Cult'' is a disconnect between form and contents: the form is
compelling, but there can't be a rational expectation of a benefit from
the activity because there is no causal connection between the activity
and the claimed outcome. This is often, but not always due to
\href{https://en.wikipedia.org/wiki/List_of_fallacies}{logical
fallacies}.

The topic is interesting for bioinformatics because the deficiencies are
often subtle, and hard for the non-expert to spot. To guard against
Cargo Cult takes integrity, and practice. A structured approach may be
helpful that first clearly identifies the hoped-for outcome, then
defines the proposed activities, then asks in specific detail how the
outcome would be caused by the activity. \textbf{Causation} is key here
- many examples of Cargo Cult behaviour are based on a mistaken belief
in causation, where actually merely a correlation was observed. But you
have to be careful: the fact that causation has not been shown does not
prove it is absent. And even if causation is absent, that does not
automatically make the behaviour invalid: sometimes you are right for
the wrong reason. Both cases are not Cargo Cult. Rather, it is
characteristic for situations we should label as Cargo Cult that there
``is no cargo in the system'': you are looking in the wrong place, you
don't have a control or reference value, you don't understand your data
- or similar problems.

\section{Task:}\label{task-4}

\begin{rmd-task}
\begin{enumerate}
\def\labelenumi{\arabic{enumi}.}
\tightlist
\item
  Read a brief introduction to ideas about
  \href{./boris_docs/FND-Cargo_Cult.pdf}{``Cargo Cult Bioinformatics''}.
\item
  Visit the
  \href{http://steipe.biochemistry.utoronto.ca/abc/students/index.php/Cargo_Cult_Science}{Student
  Wiki: Cargo\_Cult\_Science page} (and its associated
  \href{http://steipe.biochemistry.utoronto.ca/abc/students/index.php/Not_(quite)_Cargo_Cult_Science}{Student
  Wiki: Not\_(quite)\_Cargo\_Cult\_Science} page) on the Student Wiki
  and add an example of your own that you have encountered during your
  studies or elsewhere\footnote{Make sure your example has not already
    been posted by someone else - it would be Cargo Cult to post it
    again.}.
\item
  Also, add to the discusssion of any of the existing examples. Comments
  that question whether the example actually should be called Cargo
  Cult, and that lead to improved focus and clarification are especially
  valuable.
\end{enumerate}
\end{rmd-task}

\section{Self-evaluation}\label{self-evaluation-3}

\section{Further reading, links and
resources}\label{further-reading-links-and-resources-5}

\begin{itemize}
\tightlist
\item
  \href{http://steipe.biochemistry.utoronto.ca/abc/students/index.php/Cargo_Cult_Science}{Student
  Contributions: Cargo Cult Examples}
\item
  \href{http://steipe.biochemistry.utoronto.ca/abc/students/index.php/Not_(quite)_Cargo_Cult_Science}{Student
  Contributions: Not (quite) Cargo Cult Examples}
\item
  \href{https://en.wikipedia.org/wiki/List_of_fallacies}{Wikipedia: List
  of Logical Fallacies} a very comprehensive resource. One would wish
  that the presence of such a list itself would have a beneficial effect
  on science.
\end{itemize}

\textbf{If in doubt, ask!} If anything about this learning unit is not
clear to you, do not proceed blindly but ask for clarification. Post
your question on the course mailing list: others are likely to have
similar problems. Or send an email to your instructor.

\BeginKnitrBlock{rmd-original-history}
\textbf{Author}: Boris Steipe
\href{mailto:boris.steipe@utoronto.ca}{\nolinkurl{boris.steipe@utoronto.ca}}
\textbf{Created}: 2017-08-05 \textbf{Modified}: 2017-09-11 Version: 1.1
\textbf{Version history}: 1.1 Add references to logical fallacies;
review older submissions and move some into Not (quite) Cargo Cult page.
1.0 Completed to first live version 0.1 Material collected from previous
tutorial
\EndKnitrBlock{rmd-original-history}

\subsection{Updated Revision history}\label{updated-revision-history-4}

\begin{tabular}{l|l|l|l}
\hline
Revision & Author & Date & Message\\
\hline
f56a24c & Ruth Isserlin & 2019-12-23 & Added new git info to each fileAdded new git info to each file (in addition to the original version history copied over from Boris's wiki).\\
\hline
8950904 & Ruth Isserlin & 2019-12-22 & Initial check in of converted wiki pages from Boris Steipe's bcb420 course material pagewiki pages were converted to bookdown and formatted to the bookdown format\\
\hline
\end{tabular}

\subsection{Footnotes:}\label{footnotes-2}

\chapter{Netiquette - (Network
Etiquette)}\label{netiquette---network-etiquette}

(Etiquette for the Internet and for this course)

\section{Overview}\label{overview-6}

\subsection{Abstract:}\label{abstract-6}

Netiquette is a portmanteau of ``network etiquette''. It is a collection
of social conventions for communication on the Internet such as e-mail,
mailing lists, forums \ldots{} Here is a subset of relevance for our
courses and workshops.

\subsection{Objectives:}\label{objectives-6}

Read about the basic rules and courtesies for communicating effectively
and professionally in mailing lists and internet forums.

\subsection{Outcomes:}\label{outcomes-6}

Apply the principles discussed on the page consistently throughout the
course. Identify weaknesses and propose updates.

\subsection{Deliverables:}\label{deliverables-6}

\textbf{Time management}: Before you begin, estimate how long it will
take you to complete this unit. Then, record in your course journal: the
number of hours you estimated, the number of hours you worked on the
unit, and the amount of time that passed between start and completion of
this unit.

\textbf{Journal}: Document your progress in your Course Journal. Some
tasks may ask you to include specific items in your journal. Don't
overlook these.

\textbf{Insights}: If you find something particularly noteworthy about
this unit, make a note in your insights! page.

\begin{itemize}
\tightlist
\item
  Is there a pet-peeve of yours that is not mentioned here? Initiate a
  discussion on that on the course mailing list or contribute to the
  discussion if one has already started.
\item
  Is this page still up to date? Cultural norms change, and our
  communication habits have changed tremendously in the past decade.
  Propose updates and/or improvements on on the course mailing list or
  contribute to the discussion if one has already started.
\end{itemize}

\begin{center}\rule{0.5\linewidth}{\linethickness}\end{center}

\section{Be kind}\label{be-kind}

This is the single most important rule. We are all working together.
Let's all make this a pleasant and exciting experience.

\section{Pay attention who you reply
to}\label{pay-attention-who-you-reply-to}

If you use ``reply'', your message will go the entire list. Pause a
moment, and consider whether this is what you want. Perhaps your message
is of interest to only a single recipient? Or your message may be
personal, or confidential \ldots{}

\section{Use informative subject
lines}\label{use-informative-subject-lines}

Spend a moment thinking what your post is about, then condense the
message into a few words. This goes a long way towards:

\begin{itemize}
\tightlist
\item
  allowing the recipients to estimate how interested they are in the
  contents of your message;
\item
  retrieving a thread in your archived messages;
\item
  browsing the archives for information;
\item
  keeping the ensuing discussion on topic.
\item
  Try to be specific for example,

  \begin{itemize}
  \tightlist
  \item
    this subject line is poor: Chimera doesn't work!
  \item
    Much better would be: Can't load molecule in Chimera after editing
    coordinates
  \end{itemize}
\item
  If you must change the subject line, quote the old line as in: New
  subject (was: old subject)
\end{itemize}

\section{Don't hijack threads}\label{dont-hijack-threads}

If you have a new question, never simply write it into the reply to an
older thread. This is called ``Thread Hijacking'' and it is rude and
ineffective and it will cause me to have a poor opinion of your manners.
If you hijack a thread:

\begin{itemize}
\tightlist
\item
  your new question will end up in an unrelated discussion;
\item
  it will be much harder to search;
\item
  the original poster's question gets diluted and may never get
  appropriately discussed;
\item
  you demonstrate that you didn't actually care enough to type up a
  subject line.
\item
  Use your mail-client's reply function if you contribute to a thread,
  write a new mail or post if you have something new to add.\footnote{Some
    mailing list software builds thread based on subject line and some
    software builds thread based on message ID. Only changing the
    subject line may not be sufficient to start a new thread. Since you
    usually don't know which does what, use the rule above to be sure.}
\item
  As a corollary: if your post is related to the thread, by all means do
  use the reply function of your mail client and don't change the
  subject line, even if you think the original subject line was not well
  written or contained a (non-critical) typo etc.
\end{itemize}

\section{Follow the discussion}\label{follow-the-discussion}

From time to time I see questions asked that have already been answered
previously. This shows me that you did not follow the discussion. What
do you think I think of that? Exactly.

\section{Describe problems clearly}\label{describe-problems-clearly}

Sometimes your problems will be due to a faulty assumption, sometimes
due to incompatible software, sometimes due to bugs, or errors in
assigned tasks \ldots{} The more clearly you describe what you did and
what happened, the more likely it is someone will be able to help.
Simply stating ``this or that didn't work'' will get you nowhere. Ask
yourself:

\begin{enumerate}
\def\labelenumi{\arabic{enumi}.}
\tightlist
\item
  Did I specify the circumstances under which the problem arose?
\item
  Did I specify exactly what happened and what I believe the problem is?
\item
  Have I given enough information so that someone else would be able to
  reproduce the problem?
\end{enumerate}

\section{Avoid screenshots}\label{avoid-screenshots}

Almost always when I see a screenshot of errors that arise during
assignments, the issue would have been better described by copy/pasting
code and text. Screenshots, as images, are a dead-ends for further
analysis.

\begin{itemize}
\tightlist
\item
  If you paste a screenshot with a sequence ID, others will need to type
  it out, awkwardly, on a different page to reproduce your problem.
\item
  If you paste a screenshot with a piece of code, others will have to
  type the code, awkwardly, into their code editor to reproduce your
  problem and experiment for a solution.
\item
  If you paste a screenshot of an error message, it is that much more
  work to Google for the message and figure out what could have caused
  it.
\end{itemize}

In all those cases, you prevent others from helping you quickly and
effectively and you are wrong to expect others to type down the contents
of your images because you didn't copy/paste the essential material.
That's not smart. Also you are wasting other's bandwidth on their
computers or mobile devices. That's rude.

The only case where screenshots are encouraged is where an image is
involved - but even then, for example when discussing R plots, the code
that has generated the plot would be more helpful. dput() is your
friend. And learn to create MWEs (Minimal Working Examples).

\section{Show us that you've done your
homework}\label{show-us-that-youve-done-your-homework}

The \href{http://www.catb.org/~esr/faqs/smart-questions.html}{``How to
Ask Questions the Smart Way''} document gives the following excellent
advice:

I can't put it any better.

\section{Use mixed case and write full
words}\label{use-mixed-case-and-write-full-words}

Using UPPERCASE ONLY IS THE TYPOGRAPHIC EQUIVALENT OF SHOUTING; this is
appropriate only under exceptional circumstances. some people use
lowercase only. are they too lazy to find the caps key?

\begin{Shaded}
\begin{Highlighting}[]
\NormalTok{It also looks lik u cant B rly botherd }\DecValTok{2}\NormalTok{ rite }
\ControlFlowTok{if}\NormalTok{ u rite ur txt with textN shorth}\OperatorTok{&}\NormalTok{.}
\end{Highlighting}
\end{Shaded}

\section{Don't Troll}\label{dont-troll}

Just don't do it, oK?

\textbf{Also remember: trolling, stalking, impersonating etc. may fall
under sanctionable offences of the
\href{https://governingcouncil.utoronto.ca/secretariat/policies/code-student-conduct-february-14-2002}{University's
Student Code of Conduct}.}

\section{No need to directly address
someone}\label{no-need-to-directly-address-someone}

Sending a message to the list reaches all list members. That's the point
of the list: we are sharing discussions with everyone. There is no need
to address anyone in particular (not even your professor), unless you
are responding publicly to a specific statement by that person. Even
generic salutations - like Howdy or Dear all, are nowadays usually
omitted for the sake of brevity.

\section{Resolve when done}\label{resolve-when-done}

Once the problem has been solved, there is no need to thank contributors
- but don't just walk away, you are not done! It is very important to
share feedback whether the advice received has worked or not, or if
something else worked instead. People helping you on a mailing list
don't expect a reward except for one thing: the little satisfaction that
their effort was actually helpful. Don't deny that, that would be rude.

A separate reason to resolving such threads is that they are archived,
and when others are looking for their solution they need to be able to
tell whether there is hope to be found here.

\section{Self-evaluation}\label{self-evaluation-4}

\textbf{Question 1}: Zachary has run into an issue when downloading an R
package. He posts a screenshot of his computer in the mailing list to
ask what could have gone wrong. Good or bad?\footnote{\textbf{Bad}.Reread
  the ``Avoid screenshots'' section above if you disagree.}

\textbf{Question 2}:Amadeus has been following a discussion on the
mailing list about Zachary's problem. That reminds him that he wasn't
sure how to find a package that contains a particular function. He
replies to Zachary's thread and asks his question. Good or
bad?\footnote{\textbf{Bad}. Reread the ``Don't hijack threads'' section
  above if you disagree.}

\textbf{Question 3}:Kaila responds to a post I have made, about some
subtleties in identifying homologous sequences by sequence alignment.
She is in a hurry and doesn't address me by name in her post. Good or
bad?\footnote{\textbf{Good}. Reread the ``No need to directly address
  someone'' section above if you disagree.}

\section{Further reading, links and
resources}\label{further-reading-links-and-resources-6}

\begin{itemize}
\tightlist
\item
  Dall'Olio et al. (2011) Ten simple rules for getting help from online
  scientific communities. PLoS Comput Biol 7:e1002202. (pmid: 21980280)
  \href{https://www.ncbi.nlm.nih.gov/pubmed/?term=21980280}{PubMed}
\item
  \href{http://www.opentextbooks.org.hk/ditatopic/4851}{Netiquette -
  Virginia Shea's Core Rules of Netiquette}
\item
  \href{https://stackoverflow.com/help/minimal-reproducible-example}{How
  to create a Minimal, Complete, and Verifiable example -- R advice via
  stackoverflow.}
\item
  \href{http://www.catb.org/~esr/faqs/smart-questions.html}{How to Ask
  Questions the Smart Way for the technical minded, excellent advice!}
\end{itemize}

\textbf{If in doubt, ask!} If anything about this learning unit is not
clear to you, do not proceed blindly but ask for clarification. Post
your question on the course mailing list: others are likely to have
similar problems. Or send an email to your instructor.

\BeginKnitrBlock{rmd-original-history}
\textbf{Author}: Boris Steipe
\href{mailto:boris.steipe@utoronto.ca}{\nolinkurl{boris.steipe@utoronto.ca}}
\textbf{Created}: 2017-08-05 \textbf{Modified}: 2017-08-17 Version: 1.1
\textbf{Version history}: 1.1 Add thread resolution (Gabi Morgenshterns
suggestion). 1.0 First live version using contents from an old page. 0.1
First stub
\EndKnitrBlock{rmd-original-history}

\subsection{Updated Revision history}\label{updated-revision-history-5}

\begin{tabular}{l|l|l|l}
\hline
Revision & Author & Date & Message\\
\hline
f56a24c & Ruth Isserlin & 2019-12-23 & Added new git info to each fileAdded new git info to each file (in addition to the original version history copied over from Boris's wiki).\\
\hline
8950904 & Ruth Isserlin & 2019-12-22 & Initial check in of converted wiki pages from Boris Steipe's bcb420 course material pagewiki pages were converted to bookdown and formatted to the bookdown format\\
\hline
\end{tabular}

\subsection{Footnotes:}\label{footnotes-3}

\chapter{Information Sources for
Bioinformatics}\label{information-sources-for-bioinformatics}

(Wikipedia, NAR, Bioinformatics.ca, PubMed, Citation index)

\section{Overview}\label{overview-7}

\subsection{Abstract:}\label{abstract-7}

This unit introduces key information sources for bioinformatics:
journals, forums, and supporting sites.

\subsection{Deliverables:}\label{deliverables-7}

\textbf{Time management}: Before you begin, estimate how long it will
take you to complete this unit. Then, record in your course journal: the
number of hours you estimated, the number of hours you worked on the
unit, and the amount of time that passed between start and completion of
this unit.

\textbf{Journal}: Document your progress in your Course Journal. Some
tasks may ask you to include specific items in your journal. Don't
overlook these.

\textbf{Insights}: If you find something particularly noteworthy about
this unit, make a note in your insights! page.

\begin{rmd-task}
\begin{itemize}
\tightlist
\item
  Read the \href{boris_docs/FND-BIN-Concepts.pdf}{introductory notes on
  bioinformatics concepts} for some general observations on
  bioinformatics, to set the stage.
\item
  Read the \href{boris_docs/BIN-Info_sources.pdf}{introductory notes on
  information sources for bioinformatics}.
\end{itemize}
\end{rmd-task}

\section{Journals}\label{journals}

\begin{rmd-task}
\begin{itemize}
\tightlist
\item
  Visit the \href{https://academic.oup.com/nar}{Nucleic Acids Research
  Journal (NAR) site} and find the current database volume and the Web
  service volume.
\item
  Task yourself to find at least one database and service that interests
  you, visit it and poke around. You should aim to develop an intuition
  for what to expect with such resources and how to use the services.
\item
  -- (Next, visit bioinformatics.ca, navigate to the ``Bioinformatics
  links directory'' and try a sample search - e.g.~for ``disorder'', or
  ``localization''. This service has been discontinued - thanks for
  notifying me, Xiaowen \ldots{} will update.) --
\end{itemize}
\end{rmd-task}

Incidentally: you can subscribe to regular Table of Contents updates
from any journal. \href{https://www.nature.com/}{nature} and
\href{https://science.sciencemag.org/}{Science} should for sure be in
your inbox, but subscribe to some of the bioinformatics journal alerts
too, at least for this term. A current list of journals is here:
\href{https://en.wikipedia.org/wiki/List_of_bioinformatics_journals}{List
of bioinformatics journals}.

\section{Forums}\label{forums}

Much current, active exchange of bioinformatics knowledge happens in
non-traditional platforms:

\begin{rmd-task}
Visit each of the forums below and find (at least) one item that
interests you.

\begin{itemize}
\tightlist
\item
  \href{https://www.biostars.org/}{BioStars}: General bioinformatics,
  computational-, and systems biology questions (timesink warning!)
\item
  \href{https://www.reddit.com/r/bioinformatics/}{Reddit}: the
  bioinformatics ``subreddit'' (timesink warning!)
\item
  \href{https://stat.ethz.ch/mailman/listinfo/r-help}{R-help}: The R
  programming language
\item
  \href{https://stackoverflow.com/questions/tagged/r}{Stack Overflow}:
  R-related questions
\item
  \href{https://www.bioconductor.org/help/support/}{BioConductor
  Support}: for all questions about the BioConductor Project
\item
  \href{https://stats.stackexchange.com/}{Cross Validated}: statistics
  related questions on Stack-exchange
\end{itemize}
\end{rmd-task}

\section{Self-evaluation}\label{self-evaluation-5}

\section{Further reading, links and
resources}\label{further-reading-links-and-resources-7}

\textbf{If in doubt, ask!} If anything about this learning unit is not
clear to you, do not proceed blindly but ask for clarification. Post
your question on the course mailing list: others are likely to have
similar problems. Or send an email to your instructor.

\BeginKnitrBlock{rmd-original-history}
\textbf{Author}: Boris Steipe
\href{mailto:boris.steipe@utoronto.ca}{\nolinkurl{boris.steipe@utoronto.ca}}
\textbf{Created}: 2017-08-05 \textbf{Modified}: 2019-09-14 Version: 1.1
\textbf{Version history}: 1.1 Bioinformatics.ca links database no longer
exists 1.0 Live version 0.1 First stub
\EndKnitrBlock{rmd-original-history}

\subsection{Updated Revision history}\label{updated-revision-history-6}

\begin{tabular}{l|l|l|l}
\hline
Revision & Author & Date & Message\\
\hline
f56a24c & Ruth Isserlin & 2019-12-23 & Added new git info to each fileAdded new git info to each file (in addition to the original version history copied over from Boris's wiki).\\
\hline
8950904 & Ruth Isserlin & 2019-12-22 & Initial check in of converted wiki pages from Boris Steipe's bcb420 course material pagewiki pages were converted to bookdown and formatted to the bookdown format\\
\hline
\end{tabular}


\end{document}
